\chapter{Algorithms to compute the Galois group of a rational polynomial}
\label{cha:algor-comp-galo}


We will now deal with the problem  to compute the Galois group of a polynomial $f(x) ∈ ℚ[x]$, i.e., $\gal(E:ℚ)$, where $E$ is a splitting field of $f$. We will proceed as follows.

\begin{enumerate}[i)]
\item We will compute the minimal polynomial $p_α(x) ∈ ℚ[x]$ of a primitive element $α$ of $E$, i.e. the complex number $α$ with $ℚ(α) = E$.
\item Let $α= α_0, α_1,\dots,α_{n-1}$ be the roots of $p(x)$.  We will then compute the matrices $A_i ∈ℚ^{n ×n}$,  such that
  \begin{displaymath}
    \begin{pmatrix}
      α_{i}^{0} \\
      α_{i}^{1} \\
      \vdots \\
       α_{i}^{n-1}       
    \end{pmatrix} = A_i \begin{pmatrix}
      α_{0}^{0} \\
      α_{0}^{1} \\
      \vdots \\
       α_{0}^{n-1}       
    \end{pmatrix}
  \end{displaymath} holds.
\item With the list of matrices $A_i$ t hand, we can then completely describe the group table of the Galois group. 
\end{enumerate}


Our aim is to design an algorithm that is polynomial in the binary encoding length of the input polynomial $f(x) ∈ ℚ[x]$  and the cardinality of the Galois Group. 

\subsubsection*{Exercises}

\begin{enumerate}
\item How does the first row of $A_i$ look like? Justify your answer.
\item Let $σ ∈ \gal(E:ℚ)$ with $σ(α) = α_i$, then $σ$ is also an isomorphism of the $ℚ$-vector space $E$. What is the matrix of the $σ$ with respect to the basis  $1,α,\dots,α^n$? 
\item Explain how to compute the other rows of $A_i$ quickly with the following input:
  \begin{enumerate}[a)] 
  \item The second row of $A_i$
  \item The minimal polynomial $p(x) ∈ℚ[x]$ of $α = α_0$. 
  \end{enumerate}
\item Let $σ_i ∈ \gal(E:ℚ)$ be the $ℚ$-automorphism of $E$ with $σ(α_0) = α_i$. Then $σ_i:E → E$ is also an automorphism of the $ℚ$-vector space $E$. An element $e ∈E$ is a linear combination of the basis elements $α^0_0,α^1_0,\dots,α^{n-1}_0$. If $φ:E →ℚ^n$ is the coordinate isomorphism with $φ(x_0 α^0_0+x_1 α_1^0+\cdots+ x_{n-1}α^{n-1}_0) =(x_0,\dots,x_{n-1})^T $, then which matrix makes the following diagram commute:
  \begin{displaymath}
    \begin{CD}
      E @>σ_i>> E \\
      @VVφV  @VVφV \\
      ℚ^n @> A x >> ℚ^n
    \end{CD}
  \end{displaymath}
\item Describe how to fill in the group table that describes $\gal(E:ℚ)$, once the matrices $A_i$ are computed. 
\end{enumerate}

\section{Retrieving the minimal polynomial of an algebraic number}
\label{sec:retr-minim-polyn}

We deal with the problem of identifying the minimal polynomial $p(x) ∈ℤ[x]$ of an algebraic number $α ∈ℂ$, provided that we are given:
\begin{enumerate}[i)]
\item A sufficiently good approximation $\br{α} ∈ℂ$  of $α$.\label{item:18}
\item An upper bound on the norm $\|p\|_2$ of $p(x)$. \label{item:19}
\item An upper bound on the degree of $p(x)$. 
\end{enumerate}



We can assume that $|α|≤1$, since we can otherwise retrieve the minimal polynomial of $1/α$ instead, see Exercise~\ref{item:20}) below.  We will show the following result.

\begin{theorem}[(Kannan, Lovàsz and  Lenstra \cite{kannan1988polynomial})] \label{thr:35}
  Let $α ∈ℂ$ be an algebraic number with minimal polymomial $p(x) ∈ℤ[x]$ of degree $n$. Provided that one is given  approximations $\br{α}_i$ $i=0,\dots,n$  of $α^i$ that satisfy
  \begin{displaymath}
    | \br{α}_i - α^i| < 1 / 2^{c (n^2 + n \log \|p\|_∞)} ,  
  \end{displaymath}
  where $c>1$ is an absolute constant, then one can find $p(x)$ in time polynomial in $n$ and $\log \|p\|_∞$. 
\end{theorem}

Let us discuss why this algorithm enables us to factor polynomials in $ℚ[x]$. Let $f(x)∈ℤ[x]$ be a polynomial of degree $n$ and content one. We would like to be able to decide whether $f(x)$ is irreducible and if not, find an irreducible factor $p(x) ∈ℤ[x]$ of $f(x)$. Theorem~\ref{thr:7} shows that any irreducible factor $p(x) ∈ℤ[x]$ of $f(x)$ satisfies
\begin{displaymath}
\|p\|_∞ ≤  \|p\|_1 ≤ 2^n  ⋅ \|f\|_2≤ 2^{2 n} \|f\|_∞. 
\end{displaymath}
With variants of Newtons method, we can efficiently compute an approximation of a root $α ∈ℂ$ and complex numbers  $\br{a}_i∈ℂ$ with 
 \begin{displaymath}
    | \br{α}_i - α^i| < 1 / 2^{c (n^2 + n \log(2^{2 n} \|f\|_∞)  )} ≤ 1 / 2^{c' (n^3 + n \log( \|f\|_∞)  )},
    \end{displaymath}
    where $c'$ is an absolute constant that only depends on $c$. A course on numerical methods treats algorithms that make this computation in the desired running time.  Now we can use Theorem~\ref{thr:35} and compute the minimal polynomial of $α$ efficiently. This minimal polynomial is either $f(x)$, in which case $f$ is irreducible, or a nontrivial factor of $f(x)$. A corollary is the celebrated result of Lenstra, Lenstra and Lovàsz~\cite{lenstra1982factoring}.

    \begin{corollary}[\cite{lenstra1982factoring}]
      \label{co:3}      
      There exists a polynomial time algorithm that, given a polynomial $f(x) ∈ℤ[x]$ of content one, decides whether $f(x)$ is irreducible and in the case in which it is not, finds a nontrivial factor $p(x) ∈ℤ[x]$ of $f(x)$. 
    \end{corollary}


\subsubsection*{Exercises}

\begin{enumerate}
\item Show that if $α ∈ℂ \setminus \{0\}$ is a root of $f(x) = a_0 + a_1x + \cdots + a_n x^n ∈ℂ[x]$, then $1/α$ is a root of $\wt{f}(x) = a_n + a_{n-1}x + \cdots + a_0x^n$.  Also if $f(x)$ can be factored into $f(x) = p(x) q(x)$, then $\wt{f}(x) = \wt{p}(x) \wt{q}(x)$, where $\wt{p}(x)$ and $\wt{q}(x)$ are analogously defined. \label{item:20}

\item Let $α = (a + ib)∈ℂ$ be a complex number with $|α|≤1$, the binary expansion of $a$ and $b$ being $a = ∑_{i=1}^∞ a_i 2^{-i}$ and  $b = ∑_{i=1}^∞ b_i 2^{-i}$, where the $a_i, b_i ∈ \{0,1\}$. The numbers $a_1,\dots,a_k$ and $b_1,\dots,b_k$ are called the first $k$ bits of the real and imaginary part of $α$ respectively. In the following, let $\br{α} ∈ℂ$ be a complex number of absolute value at most one. 
  \begin{enumerate}[a)]
  \item If the first $k$ bits of the real and imaginary parts of $α$ and $\br{α}$ coincide, then $|α - \br{α}| ≤2^{-(k-1)}$.
  \item If the first $k+1$ bits of the real and imaginary parts of $α$ and $\br{α}$ coincide, then $|α^2 - \br{α}^2| ≤2^{-(k-2)}$.
  \item If the first $k+1$ bits of the real and imaginary parts of $α$ and $\br{α}$ coincide, then $|α^j - \br{α}^j| ≤2^{-(k-1)} j$.
  \item Conclude that, $\br{α}_i$ in Theorem~\ref{thr:35} can be set to $\br{α}_1^i$, if the real and imaginary parts of $\br{α_1}$ coincide with those of $α$ in the first $c' (n^3 + n^2 \log \|p\|_∞)$ bits, where $c'$ is an absolute constant, depending only on $c$. 
  \end{enumerate}
\end{enumerate}



\subsection{Root separation}
\label{sec:root-separation}

If $p(x) ∈ℤ[x]$ is an irreducible polynomial and $α∈ℂ$ is a root of $p(x)$, then we would like to have good bounds on $|q(α)|$, where $q(x) ∈ℤ[x]$ is a polynomial of degree at most $\deg(p)$ which is not a multiple of $p(x)$, i.e., of  which $α$ is not a root. A good bound is given in the next lemma which makes use of the \emph{resultant}. 


\begin{lemma}[Root separation]\label{lem:10}\
  Let $p(x), q(x) ∈ℤ[x]$ be polynomials with $\deg(p) = n$ and $\deg(q) ≤ n$. If $α∈ℂ$ is a root of $p(x)$ but not a root of $q(x)$  and if $p(x)$ is irreducible, then
  \begin{displaymath}
    | q(α)| ≥ n^{-1} \|p\|^{-n} \|q\|^{-n}. 
  \end{displaymath}
\end{lemma}

\begin{proof}
  Let $p(x) = a_0 + a_1 x + \cdots + a_n x^n$ and $q(x) = b_0+b_1x+ \cdots+ b_nx^n$. We consider the $2n × 2n$ integer matrix
  \begin{displaymath}
    A = 
    \begin{pmatrix}
      a_0  &   0   &     &    &  b_0  &   0   &     &\\
      a_1  &   a_0 &     &    & b_1  &   b_0 &     &  \\
      \vdots &  a_1&  \ddots &  & \vdots &  b_1&  \ddots &   \\
              & \vdots&       & a_0 &  & \vdots&       & b_0 \\
              a_n    &       &       & a_1  &b_n    &       &       & b_1  \\
              \hline
               &   a_n  & &    &  &   b_n  & & \\
               &       & \ddots & &  &       & \ddots &\\
               &       &        & a_n & &       &        & b_n 
    \end{pmatrix}
  \end{displaymath}
  The determinant $\det(A)$ of $A$ is called the \emph{resultant} of $A$. If the resultant was zero, then there would be a nonzero element
  \begin{displaymath}
    \begin{pmatrix}
      g_0\\
      g_1\\
      \vdots\\
      g_{n-1} \\
       h_0\\
      h_1\\
      \vdots\\
      h_{n-1} 
    \end{pmatrix}
  \end{displaymath}
  in the kernel of $A$. With the polynomials $g(x) = g_0 + \cdots + g_{n-1}x^{n-1}$ and $h(x) = h_0 + \cdots + h_{n-1}x^{n-1}$ this implies
  \begin{displaymath}
    p(x) g(x) = - q(x) h(x),
  \end{displaymath}
  with nonzero polynomials $g(x)$ and $h(x)$ of degree at most $n-1$. But then $p(x)$ divides $h(x)$, as $q(x)$ is not a multiple of $p(x)$. This is a contradiction as $p(x)$ is of degree $n$ and $h(x)$ is of degree at most $n-1$.


  We next multiply the $i$-th row with $α^{i-1}$ and add the result  to the first row. This is an elementary operation that does not change the determinant.  The new first row the reads
  \begin{displaymath}
    \begin{matrix}
          p(α)  & α p(α) & \cdots & α^{n-1} p(α) &  q(α)  & α q(α) & \cdots & q^{n-1} p(α). 
        \end{matrix}        
      \end{displaymath}
      and since $α$ is a root of $p(x)$ this reads as
       \begin{displaymath}
    \begin{matrix}
          0  & 0 & \cdots & 0 &  q(α)  & α q(α) & \cdots &q^{n-1} p(α). 
        \end{matrix}        
      \end{displaymath}
      Developing the determinant according to the first row, we obtain an expression
      \begin{equation}
        \label{eq:27}
        \det(A) = q(α) A_0 + αq(α) A_1 + \cdots + α^{n-1} q(α) A_{n-1}, 
      \end{equation}
      where $A_i$ is $\pm1$ times a $(2n-1) ×(2n-1)$ sub-determinant of $A$.
      From this we conclude
      \begin{equation}
        \label{eq:28}
        q(α) = \det(A) / (A_0 + α A_1 + \cdots + α^{n-1}  A_{n-1})
      \end{equation}
      and since $|α| ≤1$ and $\det(A) ∈ℤ$ and therefore of absolute value at least one, we obtain
      \begin{displaymath}
        |q(α)| ≥ 1 / ∑_{i=0}^{n-1} |A_i|. 
      \end{displaymath}
      The Hadamard bound on the determinant (see Advanced Linear Algebra 2) implies $|A_i| ≤ \|p\|^n \|q\|^n$ and therefore, we obtain the desired bound
       \begin{displaymath}
        |q(α)| ≥ 1 / (n \|p\|^n \|q\|^n) . 
      \end{displaymath}
\end{proof}



Suppose now that $p(x) ∈ℤ[x]$ is irreducible of degree $n$  with root $α∈ℂ$ with $|α|≤1$ and suppose one has approximations $ \br{α}_0 =1, \br{α}_1,\dots, \br{α}_n  ∈ℂ$ such that 
\begin{displaymath}
 | \br{α}_i - α^i| < 1/ 2^s
\end{displaymath}
for some $s ∈ℕ$. Then with $p(x) = a_0 + a_1x + \cdots + a_n x^n$ one has
\begin{equation}
  \label{eq:29}
  \begin{array}{rcl}
    | a_0 + a_1 \br{α}_1 + \cdots + a_n \br{α}_n|  & = &   | a_1 (\br{α}_1 - α) + \cdots + a_n (\br{α}_n - α^n) |\\
                                                 & ≤ & n ⋅ \|p\|_∞ / 2^s.
  \end{array}
\end{equation}
%
On the other hand, if $q(x) = b_0+ b_1 x + \cdots + b_n x^n ∈ℤ[x]$ is a polynomial of degree at most $n$ and of which $α$ is not a root, then the root separation Lemma~\ref{lem:10} implies
\begin{equation}
  \label{eq:30}
  \begin{array}{rcl}
    |b_0+b_1 \br{α}_1 + \cdots +b_n \br{α}_n|  & = & |b_0+b_1 α + \cdots +b_n α^n + b_1 (\br{α}_1-α) + \cdots +b_n (\br{α}_n -  α^n)| \\
                                               & ≥ & |b_0+b_1 α + \cdots +b_n α^n+ - n \|q\|_∞ / 2^s \\
    & ≥ & 1/ (n \|p\|^n \|q\|^n) - n \|q\|_∞ / 2^s. 
  \end{array}
\end{equation}

For a reason that will become clear later, we will only need to worry about polynomials $q(x)$ with $\|q\| ≤2^n \|p\|$. In this case, the bound~\eqref{eq:30} becomes
\begin{equation}
  \label{eq:31}
  \begin{array}{rcl}
    
    |b_0+b_1 \br{α}_1 + \cdots +b_n \br{α}_n| & ≥ &1/ (n 2^n \|p\|^{2n}) - n \|q\|_∞ / 2^{s} \\
    & ≥ &  1/ (n 2^n \|p\|^{2n}) -  n \|p\|/ 2^{s-n}.
  \end{array}                                          
\end{equation}
We summarize this as follows.  If
\begin{enumerate}[i)]
\item $\deg(q) ≤n$,   
\item $2^s ≥ 2n^2 2^{2n} \|p\|^{2n+1}$, 
\item $q(α) ≠0$, and
\item $\|q\| ≤ 2^n \|p\|$, 
\end{enumerate}
then the bound \eqref{eq:31} implies 
\begin{equation}
  \label{eq:32}
  |b_0+b_1 \br{α}_1 + \cdots +b_n \br{α}_n|  ≥ 1/ (2 n 2^n \|p\|^{2n}). 
\end{equation}
On the other hand, if $2^s ≥ 2n^2 2^{2n} \|p\|^{2n +1}$, then the bound~\eqref{eq:29} implies
\begin{equation}
  \label{eq:33}
   | a_0 + a_1 \br{α}_1 + \cdots + a_n \br{α}_n| ≤ 1/ (2 n 2^{2n} \|p\|^{2n}) 
 \end{equation}
 which is a $2^n$ factor smaller than the lower bound~\eqref{eq:32}.
 We have shown the following.

 \begin{theorem}
   \label{thr:36}
   Let $p(x) =a_0+a_1x+\cdots+a_nx^n∈ℤ[x]$ be an irreducible polynomial of degree $n$ and $α ∈ℂ$ be a root of $p(x)$ with $|α|≤1$.  Let $q(x) = b_0+b_1x+\cdots+b_nx^n ∈ ℤ[x]$ of degree at most $n$ with $\|q\| ≤ 2^n \|p\|$ and suppose that $α$ is not a root of $q(x)$.  Given approximations  $ \br{α}_0 =1, \br{α}_1,\dots, \br{α}_n  ∈ℂ$ such that 
\begin{displaymath}
 | \br{α}_i - α^i| < 1/ 2^s,
\end{displaymath}
where $2^s ≥ 2n^2 2^{2n} \|p\|^{2n +1}$, one has
\begin{equation}
  \label{eq:35}
  | a_0 + a_1 \br{α}_1 + \cdots + a_n \br{α}_n| ≤ 1/ (2 n 2^{2n} \|p\|^{2n}) 
\end{equation}
and
\begin{equation}
  \label{eq:36}
 |b_0+b_1 \br{α}_1 + \cdots +b_n \br{α}_n|  ≥ 1/ (2 n 2^n \|p\|^{2n}). 
\end{equation}
\end{theorem}


 \subsection{A corresponding lattice problem}
 \label{sec:corr-latt-probl}
 In the following, we use the notation of Theorem~\ref{thr:36}. 
 Let $\br{α}_i = r_i + i \ell_i$ with $r_i,\ell_i ∈ℚ$ and consider the rational matrix
 \begin{displaymath}
  B =  \begin{pmatrix}
     M ⋅ r_0 & M ⋅ r_1 & \cdots & M ⋅r_n \\
     M ⋅ \ell_0 & M ⋅ \ell_1 & \cdots & M ⋅\ell_n \\
     1 & \\
     0 & 1 \\
     &  & \ddots \\
     & & & 1
   \end{pmatrix}
 \end{displaymath}
 where $M =  \|p\| ⋅ 2 n 2^{2n} \|p\|^{2n}$. Let 
 \begin{displaymath}
   v = B    \begin{pmatrix}
     b_0 \\ \vdots \\ b_n
   \end{pmatrix}
 \end{displaymath}
 where
 \begin{displaymath}
     \begin{pmatrix}
     b_0 \\ \vdots \\ b_n
   \end{pmatrix} ∈ℤ^{n+1}
 \end{displaymath}
  can be understood as the integer vector containing the coefficients of an integer polynomial  $q(x)$. We compute 
  \begin{displaymath}
   \|v\|^2 = M^2 ⋅ |b_0+b_1 \br{α}_1 + \cdots +b_n \br{α}_n|^2 + \|q\|^2.     
 \end{displaymath}
 If $q(x) = p(x)$, then
 \begin{equation}
   \label{eq:37}
   \|v\|^2 ≤\|p\|^2 + \|p\|^2 = 2 \|p\|^2. 
 \end{equation}
 If $q(x)$ is not a multiple of $p(x)$ and $\|q\| ≤ 2^{n} \|p\|$, then Theorem~\ref{thr:36} implies
 \begin{equation}
   \label{eq:38}
   \|v\|^2 ≥ 2^{2n} \|p\|^2 + \|q\|^2 > 2^{2n} \|p\|^2.  
 \end{equation}
 And if $q(x)$ is a polynomial with $\|q\|> 2^n \|p\|$, then 
 \begin{equation}
   \label{eq:39}
    \|v\|^2 > 2^{2n} \|p\|^2.
 \end{equation}
 
 We will next see that there exists an efficient algorithm, that will compute a nonzero integer vector $(a_0,a_1,\dots,a_n)^T$ such that
$   \|v\| $ is at most $2^{n/2}$ times the minimum norm that can possibly be attained. Since $2^n ⋅2 \|p\|^2 <  2^{2n} \|p\|^2$ for $n>1$, the corresponding polynomial must be an integer  multiple of $p(x)$, which means that we can efficiently retrieve the minimal polynomial of $α$. 
 

 
%%% Local Variables:
%%% mode: latex
%%% TeX-master: "notes"
%%% End:

\chapter{Algorithms to compute the Galois group of a rational polynomial}
\label{cha:algor-comp-galo}


We will now deal with the problem  to compute the Galois group of a polynomial $f(x) ∈ ℚ[x]$, i.e., $\gal(E:ℚ)$, where $E$ is a splitting field of $f$. We will proceed as follows.

\begin{enumerate}[i)]
\item We will compute the minimal polynomial $p_α(x) ∈ ℚ[x]$ of a primitive element $α$ of $E$, i.e. the complex number $α$ with $ℚ(α) = E$.
\item Let $α= α_0, α_1,\dots,α_{n-1}$ be the roots of $p(x)$.  We will then compute the matrices $A_i ∈ℚ^{n ×n}$,  such that
  \begin{displaymath}
    \begin{pmatrix}
      α_{i}^{0} \\
      α_{i}^{1} \\
      \vdots \\
       α_{i}^{n-1}       
    \end{pmatrix} = A_i \begin{pmatrix}
      α_{0}^{0} \\
      α_{0}^{1} \\
      \vdots \\
       α_{0}^{n-1}       
    \end{pmatrix}
  \end{displaymath} holds.
\item With the list of matrices $A_i$ t hand, we can then completely describe the group table of the Galois group. 
\end{enumerate}


Our aim is to design an algorithm that is polynomial in the binary encoding length of the input polynomial $f(x) ∈ ℚ[x]$  and the cardinality of the Galois Group. 

\subsubsection*{Exercises}

\begin{enumerate}
\item How does the first row of $A_i$ look like? Justify your answer.
\item Explain how to compute the other rows of $A_i$ quickly with the following input:
  \begin{enumerate}[a)] 
  \item The second row of $A_i$
  \item The minimal polynomial $p(x) ∈ℚ[x]$ of $α = α_0$. 
  \end{enumerate}

\end{enumerate}

%%% Local Variables:
%%% mode: latex
%%% TeX-master: "notes"
%%% End:

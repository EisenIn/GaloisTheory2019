\chapter{Galois Groups and Galois Extensions}
\label{cha:galois-groups-galois}

We now come to a central definition of this course. 

\begin{definition}
  \label{def:2}
  If $E ⊇F$ are fields, an automorphism $σ: E → E$ that satisfies $σ(a) = a$ for each $a ∈ F$ is said to \emph{fix} $F$ and it is called an \emph{$F$-automorphism}.  The set of $F$-automorphisms are, together with composition a group. It is called the \emph{Galois group} of the extension $E ⊇F$ and we denote it by $\gal(E:F)$.
\end{definition}

The next lemma is simple to prove, but very helpful in understanding the Galois group.

\begin{lemma}
  \label{lem:2}
  Let $E ⊇F$  be a field extension, $f(x) ∈ F[x]$, $τ ∈ \gal(E:F)$ and $u ∈ E$. Then $τ(f(u)) = f(τ(u))$. In particular, if $u$ is a root of $f$, then so is $τ(u)$. 
\end{lemma}
\begin{proof}
  Let $f(x)  = a_0 + a_1x + \cdots + a_n x^n ∈F[x]$, then
  \begin{eqnarray*}
    τ(f(u)) & = & τ (a_0 + a_1u + \cdots + a_n u^n) \\
            & = & τ (a_0) + τ(a_1u)  + \cdots + τ(a_n u^n) \\
            & = & τ (a_0) + τ(a_1)τ(u)  + \cdots + τ(a_n) τ (u)^n) \\
            & = &  a_0 + a_1τ(u)  + \cdots + a_n τ (u)^n),
  \end{eqnarray*}
  where the  second and third row follow  from the homomorphism property of $τ$ and the last row from the fact that $τ$ fixes $F$. 
\end{proof}

\begin{theorem}
  \label{thr:17}
  Let $F(u) ⊇F$ be a simple and finite extension, let $p(x) ∈ F[x]$ be the minimal polynomial of $u$ and let $u=u_1, u_2,\dots,u_k$ be the other roots of $p(x)$ that are contained in $F(u)$. Then
  \begin{displaymath}
    \gal(F(u) :F) = \{ σ_1,\dots,σ_k\} 
  \end{displaymath}
  where $σ_i: F(u) → F(u)$ is the unique $F$-automorphism that maps $u$ to $u_i$. 
\end{theorem}

\begin{proof}
   Suppose that the degree of $p(x)$ is $n$.
  
  Let $τ ∈ \gal(E:F)$ be given. Lemma~\ref{lem:2} shows that $τ(u) = u_i$ for some $i∈ \{1,\dots,k\}$ must hold. Also, each element of $F(u)$ is of the form $f(u)$ with $f(x) ∈ F[x]$. This shows that $τ$ is uniquely specified by the image of $u$, which is another root of $p(x)$ in $F(u)$.

  On the other hand, $F(u) = F(u_i)$ for each $i=1,\dots,k$. An element of $F(u)$ is of the form $a_0+ a_1 u + \cdots + a_{n-1} u^{n-1}$ with $a_i ∈F$. The mapping  $τ_i:F(u)→F(u)$ with $a_0+ a_1 u + \cdots + a_{n-1} u^{n-1} ↦ a_0+ a_1 u_i + \cdots + a_{n-1} u_i^{n-1}$  is an $F$-automorphism which maps $u$ to $u_i$. 
\end{proof}




Equipped with this knowledge, we can start determining some Galois groups. 

\begin{example}
  \label{exe:2}
  Let us determine $\gal(ℂ:ℝ)$. Since $ℂ = ℝ(i)$ and $-i$ is the other
  root of the minimal polynomial $x^2+1$ of $i$, we understand that
  $\gal(ℂ:ℝ) =\{ id, γ\}$, where $γ$ is the complex conjugation
  automorphism.
\end{example}


\begin{example}
  \label{exe:3}
  Let $u = \sqrt[3]{2}$ and consider the extension $ℚ(u) ⊇ ℚ$. The minimal polynomial of $u$ is $x^3 -2$. The other roots of this polynomial are $w u$ and $w^2 u$ with $w =e^{2πi/3}$. These other roots are not in $ℝ$. Therefore, we conclude that $\gal( ℚ(u) : ℚ) = \{id \}$. 
\end{example}


\begin{example}
  \label{exe:4}
  Let $ u = e^{2πi/p}$, where $p$ is a prime number and consider $ℚ(u) ⊇ℚ$. The minimal polynomial of $u$ is $1+ x + \cdots +  x^{p-1}$, the $p$-th cyclotomic polynomial. The other roots of this polynomial are $u^2,\dots,u^{p-1}$ and they are all different. Thus, by Theorem~\ref{thr:17}, $|\gal(ℚ(u):ℚ)| = p-1$. Furthermore, there exists an element $τ ∈ \gal(ℚ(u):ℚ)$ with $τ(u) = u^m$, where $m$ is a generator of $ℤ_p^*$. Clearly,  $τ^i(u) = u^{m^i} = u^{m^i \pmod{p}}$. This proves that the order of $τ$ as an element of $\gal(ℚ(u):ℚ)$ is $p-1$ and thus that $\gal(ℚ(u):ℚ) ≅ C_{p-1}$. 

\end{example}



\begin{example}
  \label{exe:5}
  Let $E = \mathrm{GF}(p^n) ⊇ ℤ_p$ be an extension of degree $n$ of $ℤ_p$. We show that $\gal(E:F) ≅ C_n$. To this end, denote   $F = ℤ_p$ and since $E$ is cyclic, we have $E = F(u)$. The minimal polynomial $p(x) ∈ F[x]$ has degree $n$ and with Theorem~\ref{thr:17} $|\gal(E:F)| ≤ n$.  Consider the Frobenius automorphism
  \begin{displaymath}
    φ(x) = x^p, \text{ for each } x ∈E and 
  \end{displaymath}
 $φ^i(x) = x^{p^i}$. Therefore, if the order of $φ$ is $k$, then each element of $E$ satisfies $x^{p^k} -x = 0$. Therefore, the order of $φ$ is $n$ and this shows that $\gal(E:F) ≅ C_n$. 
\end{example}



\begin{theorem}
  \label{thr:18}
  Let $E ⊇F$ be a finite extension, $E = F(u_1,\dots,u_n)$ and let $p_i(x) ∈ F[x]$ be the minimal polynomial of $u_i$. If $τ ∈ \gal(E:F)$, then
  \begin{enumerate}[i)] 
  \item $τ$ is uniquely determined by the choice of $τ(u_1),\dots,τ(u_n)$
  \item $τ(u_i)$ is a root of $p_i(x)$.  
  \end{enumerate}
\end{theorem}

Recall the group $D_n$. It is generated by an element $a$ of order $n$, an element $b≠a$ or order $2$ and one has $aba = b$. $D_n$ has $2n$  elements, namely 
\begin{displaymath}
  1,a,a^2,\dots,a^{n-1}, b,ba,\dots, ba^{n-1}. 
\end{displaymath}


\begin{example}
  \label{exe:6}
  Let $E$ denote the splitting field of $x^3 -2$ over $ℚ$. We show that $\gal(E:ℚ) ≅ D_3$.  We denote $\gal(E:ℚ)$ by $G$. Let $u = \sqrt[3]{2}$ and $w = e^{2πi/3}$. % Clearly, $[E:ℚ] = 6$ because $ℚ(u) ⊆ ℝ$,  $[ℚ(u):ℚ]=3$ and $6 = 3! ≥ [E:ℚ] = [E:ℚ(u)][ℚ(u):ℚ]$.

  Let $τ$ be a $ℚ$-automorphism. Theorem~\ref{thr:18} shows that $τ(u) ∈ \{ u,uw,uw^2\}$ and $τ(w) ∈ \{ w, w^2\}$ and thus that $|G| ≤ 6$. This also follows from the fact that $[E:ℚ] ≤ 3⋅2$ and that $E$ is a primitive extension. We apply Theorem~\ref{thr:13}.

\begin{displaymath}
  \begin{CD}
    E=ℚ(u,w) @>σ>> ℚ(uw,w)=E \\
    @AAA     @AAA \\
    ℚ(u) @>σ_0 >>ℚ(uw) \\
    @AAA     @AAA \\
     ℚ @>id >>ℚ
  \end{CD}
\end{displaymath}
and obtain a $ℚ$-automorphism $σ$ with $σ(u) = uw$ and $σ(w) = w$. The order of this $σ$ is three. Similarly, we can construct a $ℚ$-automoprhism $τ$ with $τ(u) = u$ and $τ(w) = w^2$. The order of $τ$ is $2$ and $στσ = τ$.  


  
\end{example}


A group $G$ of permutations on a set $X$ is to \emph{act transitively} on $X$ if for each $u,v ∈X$ there exists $τ ∈G$ with $τ(u) = v$.

\begin{theorem}
  \label{thr:19}
  Let $E ⊇F$ be the splitting field of a polynomial $f(x) ∈ F[x]$ and let $X$ denote the set of roots of $f(x)$ in $E$. Then
  \begin{enumerate}[i)] 
  \item $\gal(E:F)$ is isomorphic to a subgroup of $S_X$.\label{item:1}
  \item If $f(x)$ is irreducible, then $\gal(E:F)$ acts transitively on $X$.\label{item:2}
  \item If $f(x)$ is separable, and if $\gal(E:F)$ acts transitively on $X$, then $f(x)$ is irreducible.  \label{item:3}
  \end{enumerate}
\end{theorem}

\begin{proof}
  \ref{item:1}) is clear. Let $u,w$ be roots of $f(x)$, then $F(u) ≅F(w)$ with an $F$-automorphism that maps $u$ to $w$. By Theorem~\ref{thr:13} this can be extended to an $F$ automorphism of $E$. This implies~\ref{item:2}). To show~\ref{item:3}) we suppose that $f = gh$ with $\deg(f),\deg(g)≥1$. Let $g(u) = h(v) = 0$ for some $u,v ∈E$. Since $\gal(E:F)$ acts transitively on $X$ there exists a $τ ∈ \gal(E:F)$ with $τ(u) = v$ and thus $0=τ(g(u)) = g(τ(u))=g(v)$ which contradicts that $f$ has no repeated roots.

  
\end{proof}


\section{Normal extensions}
\label{sec:normal-extensions}



\begin{definition}
  \label{def:3}
  A field extension $E ⊇F$ is called \emph{normal extension} if every irreducible polynomial in $F[x]$ that has a root in $E$ splits into linear factors in $E[x]$.
\end{definition}


\begin{example}
  \label{exe:7}
  $ℂ⊇ℝ$ is a normal extension. The extension $ℚ(\sqrt[3]{2}) ⊇ ℚ$ is not normal, as $x^3 - 2$ does not split into linear factors.  
\end{example}


\begin{theorem}
  \label{thr:20}
  An extension $E ⊇ F$ is normal and finite if and only if it is the splitting field of a polynomial over $F$. 
\end{theorem}

\begin{figure}
  \centering
\includegraphics{normal.pdf}  
  \caption{Illustration of the proof of Theorem~\ref{thr:20}}
  \label{fig:1}
\end{figure}



\begin{proof}
  Suppose that $E ⊇ F$ is normal and finite. Then $E = F(u_1,\dots,u_k)$. Each minimal polynomial $p_i(x) ∈ F[x]$ splits into linear factors over $E$, one factor being $(x  - u_i)$. This shows that $E$ is the splitting field of $p_1(x) \cdots p_k(x)$.

  For the converse we assume that $E$ is the splitting field of $f(x) ∈F[x]$ and let $p(x) ∈ F[x]$ be an irreducible polynomial that has a root $θ_1$ in $E$.  Let $M$ be a splitting field of $p(x)$ over $E$ and let  $θ_2$ be another root of $p(x)$ in $M$. The fields $F(θ_1)$ and $F(θ_2)$ are isomorphic. The fields $E(θ_1)$ and $E(θ_2)$ are splitting fields of $f(x)$ over $F(θ_1)$ and $F(θ_2)$ respectively. Theorem~\ref{thr:13} shows that the extensions $E(θ_1)⊇F(θ_1)$ and $E(θ_2)⊇F(θ_2)$  are isomorphic and have the same degree. One has
  \begin{displaymath}
    [E(θ_1):F]  =   [E(θ_1):F(θ_1) ] ⋅ [F(θ_1) :F]  
  \end{displaymath}
  and
  \begin{displaymath}
        [E(θ_2):F]  =   [E(θ_2):F(θ_2) ] ⋅ [F(θ_2) :F]. 
  \end{displaymath}
   which proves, $ [E(θ_1):F ] = [E(θ_2):F ]$. On the other hand, one has
   \begin{displaymath}
      [E(θ_1):F ]  =   [E(θ_1):E ] ⋅ [E :F] 
    \end{displaymath}
    and
    \begin{displaymath}
      [E(θ_2):F ]  =   [E(θ_2):E] ⋅ [E :F] 
    \end{displaymath}
  which shows that $[E(θ_1):E ] =[E(θ_2):E]$ and thus that also $θ_2 ∈ E$. 
\end{proof}

Theorem~\ref{thr:20} has a very important consequence. Let $E ⊇F$ be a finite  normal and separable extension. Then, it is a primitive extension  $E = F(u)$ by Theorem~\ref{thr:16}. Furthermore, if $p(x)∈F[x]$ is the minimal polynomial of $u$, then $p(x)$ splits into linear factors in $E$
\begin{displaymath}
  p(x) = (x-u_1) \cdots (x-u_n),
\end{displaymath}
where $u_1=u$ and $u_2,\dots,u_n$ are the other roots pf $p(x)$. 
Theorem~\ref{thr:17} tells us that  $\gal(E:F) = \{ τ_1,\dots,τ_n\}$ where $τ_i$ is the unique $F$-automorphism that maps $u$ to $u_i$. Thus we fully understand the Galois group of a finite normal and separable field extensions, once we get a hold on the primitive element, its minimal polynomial and the other roots of the minimal polynomial as $F$-linear combinations of $1,u,\dots,u^{n-1}$. We will elaborate on this further. 


%%% Local Variables:
%%% mode: latex
%%% TeX-master: "notes"
%%% End:

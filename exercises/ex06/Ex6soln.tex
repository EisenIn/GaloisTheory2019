\documentclass[12pt,a4paper]{article}
\usepackage{fancyhdr}
 
\pagestyle{fancy}
\fancyhf{}
\lhead{Math-317 Galois Theory \\ Problem Set 5 - Solutions}
\rhead{October 18, 2019}
\rfoot{1 (a)-(d) by J.Baudin \\1(e),2,3,4 by S.Giampietro\\5,6 by V.A.Nadarajan}
 

\usepackage{times}
\usepackage{mathptmx}
\usepackage{mathrsfs}
\usepackage{amssymb}
\usepackage{fancyhdr}
\usepackage{amsmath, amsthm}
\usepackage{xfrac}    
\usepackage{tikz-cd}
\usepackage{enumitem}
\usepackage{commath}



\newcommand{\Q}{\mathbb{Q}}
\newcommand{\F}{\mathbb{F}}
\newcommand{\Lbar}{\overline{K}}





\begin{document}


\vskip 2ex

\textbf{Solution 1.  }\\
Recall the fact that an extension $E \supseteq F$ is Galois if and only if it is normal and separable. \\ Furthermore, recall that an extension $E \supseteq F$ is finite and normal if and only if it is a splitting field of some polynomial. \\
Throughout this exercise, the Galois group of $E \supseteq F$ will be denoted $G$.

\begin{enumerate}[label = \alph*)]
	\item Let $\zeta = e^{2i\pi / 5}$. Since $\mathbb{Q}$ is perfect, we know that is extension is separable. In addition. note that $\mathbb{Q}(\zeta)$ is a splitting field of $f(x) = x^5 - 1$ over $\mathbb{Q}$. Indeed, the roots of $f$ in $\mathbb{C}$ are the elements $\zeta^k = e^{2ik\pi / 5}$, $k = 0, 1, 2, 3, 4$ and all these roots belong to $\mathbb{Q}(\zeta)$. Hence, this extension is Galois. \\ \\
	Let us now find the Galois group of this extension. Recall the following fact : \\
	Let $p$ be a prime number, then the polynomial $g_p(x) = 1 + x + \dots + x^{p - 1}$ is irreducible over $\mathbb{Q}$ (this is proven by applying Eisenstein's criterion to the polynomial $g_p(x + 1)$). \\ \\
	Since $x^5 - 1 = (x - 1)(1 + x + x^2 + x^3 + x^4)$, the minimal polynomial of $e^{2i\pi / 5}$ is $1 + x + x^2 + x^3 + x^4 = g_5(x)$ so $[\mathbb{Q}(\zeta) : \mathbb{Q}] = 4$. Thus, $|G| = 4$. Furthermore, we know from the lecture that any element $\varphi \in G$ send $\zeta$ to an other root of $g_5$ (a.k.a $\zeta^k$, $k = 1, 2, 3, 4$) and is completely determined by this assignment. Rephrasing this, there are at most $4$ elements in $G$ (say $\varphi_k$, $1 \leq k \leq 4$) and these maps are determined by \[\varphi_k(\zeta) = \zeta^k \]
	Since we know that $|G| = 4$, these maps must be the desired automorphisms we were looking for (note that this technique allowed us not to prove that the maps $\varphi_k$ were actually field homomorphisms, which is painful). \\
	Hence, \[ G = \langle \varphi_2 \rangle \cong \mathbb{Z}/{4\mathbb{Z}} \]
	so there are exactly $3$ subgroups of $G$, i.e. $3$ intermediate field extensions beetween $E$ and $F$. Note that two of them are basically $E$ and $F$ so we just have to find the "middle" field.\\ \\
	Let $K$ be this field we are looking for. Then from the Galois correspondance, we know that $K = H^*$ where $H$ denotes the "middle" subgroup of $G$. Note that $H = \langle \varphi_4 \rangle$. Thus, \[K = \{a \in E, \varphi_4(a) = a\}\]
	As $[K : \mathbb{Q}] = 2$, any element which is not in $\mathbb{Q}$ is a primitive element for the extension $K$ so we just have to find some $a \in E \setminus \mathbb{Q}$ such that $\varphi_4(a) = a$. \\ We present you two different strategies to do it.
	
	\begin{itemize}
		\item Let \[a = a_0 + a_1\zeta + a_2\zeta^2 + a_3\zeta^3\] be a generic element of $E$ for some $a_k \in \mathbb{Q}$ (recall that $\{1, \zeta, \zeta^2, \zeta^3 \}$ is a $\mathbb{Q}$-basis of $\mathbb{Q}(\zeta)$). We just want to find $a_0, a_1, a_2, a_3 \in \mathbb{Q}$ so that $\varphi_4(a) = a$ and at least one of the $a_i \neq 0$ ($i \geq 1$). We have that \[ \varphi_4(a) = a_0 + a_1\zeta^4 + a_2\zeta^8 + a_3\zeta^{12} \] Using the fact that $\zeta^5 = 1$, $\zeta^4 = -1 - \zeta - \zeta^2 - \zeta^3$ and imposing that $\zeta(a) = a$, we obtain that \[ a_0 + a_1\zeta + a_2\zeta^2 + a_3\zeta^3 = (a_0 - a_1) - a_1\zeta + (a_3 - a_1)\zeta^2 + (a_2 - a_1)\zeta^3 \] Thus, we conclude that $a_1 = 0$ and $a_2 = a_3$ so any element of the form $a_0 + a(\zeta^2 + \zeta^3)$ works. For example, $K = \mathbb{Q}(\zeta^2 + \zeta^3)$
		\item The second strategy relies on the following observation : \[\zeta^4 = \bar{\zeta}\] where $\bar{(\cdot)}$ denotes the standard complex conjugation. Hence, we obtain that $\zeta + \zeta^4$ is fixed by $\varphi_4$ but does not belong to $\mathbb{Q}$ because $\zeta + \zeta^4 = -1 - \zeta^2 - \zeta^3$.
	\end{itemize}
	Thus, we have found that $K = \mathbb{Q}(\zeta + \zeta^4) = \mathbb{Q}(\zeta^2 + \zeta^3)$. We obtain the following lattice : \[  
	\begin{tikzcd}
	\mathbb{Q}(\zeta) \\
	\mathbb{Q}(\zeta^2 + \zeta^3) \arrow [u] \\
	\mathbb{Q} \arrow[u]
	\end{tikzcd} \]
	
	\item Let $\zeta = e^{2i\pi / 7}$. Note that in this case, the proof that the extension is Galois and the computation of the Galois Group is exactly the same as before (because $7$ is prime). \\
	Therefore, let $\varphi_k$ be the element of $G$ that sends $\zeta$ to $\zeta^k$ for $k = 1, \dots, 6$. Then \[ G = \langle \varphi_3 \rangle \cong \mathbb{Z}/6\mathbb{Z} \] Thus, there are four intermediate extensions, corresponding to the subgroups $\{id = \varphi_1\}, H_1 = \langle \varphi_2 \rangle, H_2 = \langle \varphi_6 \rangle$ and $G$. Let $K_i = H_i^*$ for $i = 1, 2$. \\ First, note that \[ [K_i : \mathbb{Q}] = |G|/|H_i| \] which is either $2$ or $3$, so $[K_i : \mathbb{Q}]$ is prime for all $i$. Hence, there are no intermediate fields beetween $\mathbb{Q}$ and $K_i$ for all $i$ (use the tower law to see this). Thus, $K_i = \mathbb{Q}(a)$ for all $a \in K_i \setminus \mathbb{Q}$. We now find such elements. \\
	As in part $a)$, note that $\zeta^6 = \bar{\zeta}$. Therefore, by the same trick we used before, \[ K_2 = \mathbb{Q}(\zeta + \zeta^6) = \mathbb{Q}(\zeta^2 + \zeta^3 + \zeta^4 + \zeta^5)\]
	To find the element for $K_1$, we use the first strategy to find, for example, that $a = \zeta + \zeta^2 + \zeta^4$ satisfies the required properties. Thus, \[ K_1 = \mathbb{Q}(\zeta + \zeta^2 + \zeta^4) \] 
	
	We obtain the following lattice : \[  
	\begin{tikzcd}
	&\mathbb{Q}(\zeta)& \\
	\mathbb{Q}(\zeta^2 + \zeta^3 + \zeta^4 + \zeta^5) \arrow[ur] & &\mathbb{Q}(\zeta + \zeta^2 + \zeta^4) \arrow[ul] \\
	&\mathbb{Q} \arrow[ul] \arrow[ur]&
	\end{tikzcd} \]
	
	\item Since $\mathbb{Q}$ is perfect, we know that this extension is separable. Furthermore, this extension is the splitting field of $f(x) = (x^2 + 1)(x^2 - 3)$. Therefore, it is Galois. \\
	Note that $[\mathbb{Q}(i, \sqrt{3}) : \mathbb{Q}] = [\mathbb{Q}(i, \sqrt{3}) : \mathbb{Q}(\sqrt{3})] [\mathbb{Q}(\sqrt{3}) : \mathbb{Q}] = 4$ (note that $[\mathbb{Q}(i, \sqrt{3}) : \mathbb{Q}(\sqrt{3})] \geq 2$ because $i \notin \mathbb{R} \supseteq \mathbb{Q}(\sqrt{3}))$. \\
	Thus, we know that $|G| = 4$. Furthermore, let $\sigma \in G$. Then $\sigma(\sqrt{3}) = \pm \sqrt{3}$ and $\sigma(i) = \pm i$. Furthermore, any map $\sigma \in G$ is completely determined by its image of $\sqrt{3}$ and $i$, so there are at most $4$ candidates. Since $|G| = 4$, the $4$ possible candidates are actually field automorphisms (otherwise $|G| < 4$...) so $G = \{id, \sigma_1, \sigma_2, \sigma_1\sigma_2\}$ where $\sigma_1$ sends fixes $i$ and sends $\sqrt{3}$ to $-\sqrt{3}$ and the opposite for $\sigma_2$. Note that \[G \cong \mathbb{Z}/2\mathbb{Z} \times \mathbb{Z}/2\mathbb{Z} \] Hence, $G$ has $5$ subgroups : $\{id\}$, $\langle \sigma_1 \rangle$, $\langle \sigma_2 \rangle$, $\langle \sigma_1\sigma_2 \rangle$ and $G$ so there are $5$ intermediate fields. \\ Furthermore, one sees that the five following intermediate fields are distincts : $\mathbb{Q}, \mathbb{Q}(i), \mathbb{Q}(\sqrt{3}), \mathbb{Q}(i\sqrt{3}), \mathbb{Q}(i, \sqrt{3})$. Hence, we have found all intermediate fields ! \\
	Finally, note that $i +\sqrt{3}$ is fixed by none of the non-trivial elements of $G$, so the subgroup corresponding to $\mathbb{Q}(i + \sqrt{3})$ must be $\{id\}$, i.e. $\mathbb{Q}(i + \sqrt{3}) = \mathbb{Q}(i, \sqrt{3})$ Galois correspondance. We obtain the following lattice : \[  
	\begin{tikzcd}
	&\mathbb{Q}(i, \sqrt{3})& \\
	\mathbb{Q}(i) \arrow[ur] & \mathbb{Q}(i\sqrt{3}) \arrow[u] &\mathbb{Q}(\sqrt{3}) \arrow[ul] \\
	&\mathbb{Q} \arrow[ul] \arrow[u] \arrow[ur]&
	\end{tikzcd} \]
	\item Since all finite fields are perfect, we know that this extension is separable. Furthermore, any finite field (say of size $p^n$) is the splitting field of $x^{p^n} - x$ over $\mathbb{F}_p$, so any field extension of finite fields is normal. Hence, we have proven that any field extension of finite fields is in fact Galois. \\
	We know compute the Galois group of this extension and to do so, as usual, we compute $[\mathbb{F}_2(u) : \mathbb{F}_2]$ to obtain the order of the Galois group $G$. \\
	We show that $p(x) = x^4 + x + 1$ is irreducible over $\mathbb{F}_2$. Note that this polynomial has no root over $\mathbb{F}_2$ so the only thing to show is that it is not divisible by any irreducible of degree $2$. We present you two strategies : 
	\begin{itemize}
		\item Suppose that an irreducible polynomial of degree $2$ divides $p$, then there exists a root of $p$ is $\mathbb{F}_4$, the field with $4$ elements. However, we know that the map $a \mapsto a^4$ fixes $\mathbb{F}_4$ so, if $a \in \mathbb{F}_4$ was a root of $p$, then \[0 = a^4 + a + 1 = a + a + 1 = 1\] which is a contradiction.
		\item Suppose that $p(x) = f_1(x)f_2(x)$. Since $p$ has no root in $\mathbb{F}_2$, then $f_i$ is an irreducible polynomial of degree $2$ for all $i$. We show that the only irreducible polynimial of degree $2$ over $\mathbb{F}_2$ is $x^2 + x + 1$ (the technique we will use is generalizable, try to get better results if you are interested !). \\
		Recall the following fact about finite fields (say of characteristic $p$) : for all $n \geq 1$, \[ x^{p^n} - x = \prod_{d|n}\mbox{irreducible polynomials of degree }d \] where the product is taken over all divisors of $n$. \\
		In our case, we get that \[x^4 - x = x^{2^2} - x = x(x - 1)(\mbox{all irreducible polynomials of degree }2)\] Comparing the degrees, we see that there is a unique irreducible polynomial of degree $2$ and one can find that it is $x^2 + x + 1$. \\
		Furthermore, we see that $x^4 + x + 1 \neq (x^2 + x + 1)^2$.
	\end{itemize}
	Thus, $|G| = [\mathbb{F}_2(u) : \mathbb{F}_2] = 4$. In particular, $\mathbb{F}_2(u) \cong \mathbb{F}_{16}$, so $G$ is a cyclic group of order $4$ generated by the Frobenius map $\sigma$, i.e. $\sigma(a) = a^2$. Then there are three intermediate fields : $\mathbb{F}_2$, $K$ and $\mathbb{F}_2(u)$ where $K = \langle \sigma^2 \rangle^*$. As we did in part $a)$, we find that the element $u + u^2$ is fixed by $\sigma^2$ and does not belong to $\mathbb{F}_2$, so $K = \mathbb{F}_2(u + u^2)$. We obtain the following lattice : \[  
	\begin{tikzcd}
	\mathbb{F}_2(u) \\
	\mathbb{F}_2(u + u^2) \arrow [u] \\
	\mathbb{F}_2 \arrow[u]
	\end{tikzcd} \]
	
\end{enumerate}

\textbf{Solution 1}\\
\begin{enumerate}[e)]
\item $\mathbb{Q}(\sqrt[4]{2}, i)\supseteq \Q$ is separable as it is a finite extension over a field of characteristic $0$. It is also normal since it is the splitting field of the polynomial $p(x)=x^4-2$. This shows that it is a Galois extension. We have shown in problem set $4$ that $G=$Gal($\mathbb{Q}(\sqrt[4]{2}, i): \Q)\sim D_4$. We know $D_4 \sim \langle r, s\ | \ r^4=1, \ s^2=1, \ srs=r^{-1}\rangle $. We then identify $r$ with the element $\sigma \in G$: $\sigma(\sqrt[4]{2})=i\sqrt[4]{2}$, $\sigma(i)=i$, $s$ with the element $\tau \in G$: $\tau(\sqrt[4]{2})=\sqrt[4]{2}$, $\tau(i)=-i$.
We will use the notations for the Galois correspondence as used in course notes. 
All the possible non trivial subgroups of $D_4$ are: 
\begin{itemize}
\item $H_1=\{\id, \sigma, \sigma^2, \sigma^3\}$. Then Gal($\mathbb{Q}(\sqrt[4]{2}, i): H_1^*)=H_1$, hence $[H_1^*:\Q]=|D_4|/|H_1|= 2$. Since this subroup fixes $i$ and $[\mathbb{Q}(i):\Q]=2$, $i$ is a primitive element, i.e. $H_1^*=\Q(i)$. 
\item $H_2=\{\id, \sigma^2\}$. Then Gal($\mathbb{Q}(\sqrt[4]{2}, i): H_2^*)=H_2$, hence $[\mathbb{Q}(i):H_2^*]=|D_4|/|H_2|= 4$. Observe that $H_2$ fixes $i + \sqrt{2}$ and this element is of degree $4$ over $\Q$, so it is a primitive element and $H_1^*=\Q(i+\sqrt{2})$. 
\item$H_3=\{\id, \tau\}$. Then Gal($\mathbb{Q}(\sqrt[4]{2}, i): H_3^*)=H_3$, hence $[H_3^*:\Q]=|D_4|/|H_3|= 4$. Since this subgroup fixes $\sqrt[4]{2}$ and $[\mathbb{Q}(\sqrt[4]{2}):\Q]=4$, $\sqrt[4]{2}$ is a primitive element, i.e. $H_3^*=\Q(\sqrt[4]{2})$.
\item $H_4 = \{\id, \tau\sigma\}$. Then Gal($\mathbb{Q}(\sqrt[4]{2}, i): H_4^*)=H_4$, hence $[H_4^*:\Q]=|D_4|/|H_4|= 4$. Since this subgroup fixes $\sqrt[4]{2} -i\sqrt[4]{2}$ and no other element of $G$ fixes it, $\sqrt[4]{2} -i\sqrt[4]{2}$ is a primitive element, i.e. $H_4^*=\Q(\sqrt[4]{2} -i\sqrt[4]{2})$.
\item $H_5=\{\id, \tau\sigma^2\}$. Then Gal($\mathbb{Q}(\sqrt[4]{2}, i): H_5^*)=H_5$, hence $[H_5^*:\Q]=|D_4|/|H_5|= 4$. Since this subgroup fixes $i\sqrt[4]{2}$ and $[\mathbb{Q}(i\sqrt[4]{2}):\Q]=4$ (minimal polynomial of $i\sqrt[4]{2}$ is also $p(x)=x^4-2$), $i\sqrt[4]{2}$ is a primitive element, i.e. $H_5^*=\Q(i\sqrt[4]{2})$.
\item $H_6=\{\id, \tau\sigma^3\}$. Then Gal($\mathbb{Q}(\sqrt[4]{2}, i): H_6^*)=H_6$, hence $[H_6^*:\Q]=|D_4|/|H_6|= 4$. Since this subgroup fixes $\sqrt[4]{2} +i\sqrt[4]{2}$ and no other element of $G$ fixes it, $\sqrt[4]{2} + i\sqrt[4]{2}$ is a primitive element, i.e. $H_6^*=\Q(\sqrt[4]{2} +i\sqrt[4]{2})$.
\item $H_7=\{\id, \sigma^2, \tau\sigma, \tau\sigma^3\}$. Then Gal($\mathbb{Q}(\sqrt[4]{2}, i): H_7^*)=H_7$, hence $[H_7^*:\Q]=|D_4|/|H_7|= 2$. Since this subgroup fixes $i\sqrt{2}$ and clearly $[\Q(i\sqrt{2}):\Q]=2$ (minimal polynomial is $x^2+2$), $i\sqrt{2}$ is a primitive element, i.e. $H_7^*=\Q(i\sqrt{2})$.
\item $H_8=\{\id, \sigma^2, \tau, \tau\sigma^2\}$. Then Gal($\mathbb{Q}(\sqrt[4]{2}, i): H_8^*)=H_8$, hence $[H_8^*:\Q]=|D_4|/|H_7|= 2$. Since this subgroup fixes $\sqrt{2}$ and clearly $[\Q(\sqrt{2}):\Q]=2$ (minimal polynomial is $x^2-2$), $\sqrt{2}$ is a primitive element, i.e. $H_8^*=\Q(\sqrt{2})$.
\end{itemize}
\end{enumerate}
Finally we can draw the lattice of intermediate fields:
\begin{equation*}
\begin{tikzcd}
& & \mathbb{Q}(\sqrt[4]{2},i) & & \\
\mathbb{Q}(\sqrt[4]{2}) \arrow [urr, "2"] & \mathbb{Q}(i\sqrt[4]{2})\arrow[ur, "2"] &\mathbb{Q}(\sqrt{2} + i)\arrow[u, "2"] & \mathbb{Q}(\sqrt[4]{2} - i\sqrt[4]{2})\arrow[ul, "2"] & \mathbb{Q}(\sqrt[4]{2} + i\sqrt[4]{2})\arrow[ull, "2"]\\
& \mathbb{Q}(\sqrt{2}) \arrow[ul, "2"]\arrow[u,"2"] \arrow[ur,"2"]&  \mathbb{Q}(i)\arrow[u, "2"] & \mathbb{Q}(i\sqrt{2})\arrow[ul, "2"]\arrow[u, "2"]\arrow[ur, "2"] &\\
& & \Q \arrow[ul, "2"] \arrow[u, "2"] \arrow[ur, "2"]& &
\end{tikzcd}
\end{equation*}
Where the inclusions are the reverse inclusions of the lattice of subgroups of $D_4$:
\begin{equation*}
\begin{tikzcd}
& & \{\id\} \arrow [drr] \arrow [dr] \arrow [d] \arrow [dl] \arrow [dll] & & \\
\{\id, \tau\} \arrow[dr]& \id \{\tau\sigma^2\} \arrow[d] & \{\id, \sigma^2\} \arrow[dl] \arrow[d]\arrow[dr]& \{\id, \tau\sigma\} \arrow[d] & \{\id, \tau\sigma^3\} \arrow[dl] \\
& \{\id, \sigma^2, \tau, \tau\sigma^2 \} \arrow[dr] & \{\id, \sigma, \sigma^2, \sigma^3 \} \arrow[d]  &\{\id, \sigma^2, \tau\sigma , \tau\sigma^3 \} \arrow[dl] &\\
& & D_4 & &
\end{tikzcd}
\end{equation*}
\textit{Remark:} Observe that if two subgroups of the Galois group are conjugate to one other, then the corresponding subfields are conjugate over $\Q$ (the primitive elements of the corresponding subfields are roots of the same minimal polynomial).  In $D_4$, the subgroups $\{\id, \tau\}$ and  $\{\tau\sigma^2\}$ are conjugate, and indeed we have that the corresponding subfields $\Q(\sqrt[4]{2})$ and $\Q(i\sqrt[4]{2})$ are conjugate. Similarly $ \{\id, \tau\sigma\}$ and $\{\id, \tau\sigma^3\}$ are conjugate, and indeed so are the corresponding subfields $Q(\sqrt[4]{2} -i\sqrt[4]{2})$ and $\Q(\sqrt[4]{2} +i\sqrt[4]{2})$.\\
\\
\\
\textbf{Solution 2} \\
\\
You have shown in class that for a finite Galois extension $E\supseteq F$, there is a bijection between the intermediate fields $E\supseteq K\supseteq F$ and the subgroups $H$ of $G$.  Let $G$ be the Galois group of $E\supseteq F$. Since $G$ is finite, because $[E:F]$ finite, it has a finite number of subgroups, hence there are also a finite number of intermediate fields. 
\\
\\
\textbf{Solution 3}\\
\\
Let $E\supseteq F$ a field extension and $\mathcal{F}, \mathcal{H}$ as defined in lecture notes. In the following, $H,I\in \mathcal{H}$ and $K,L\in \mathcal{F}$. 
\begin{enumerate}[i)] 
\item By definition, $H^*= E_H=\{x\in E \ | \ \tau(x)=x \ \forall \tau \in H\}$ and $H^{**}=$Gal$(E:H^{*})=\{ \tau \in \text{Gal}(E:F) \ | \ \tau(x)=x \ \forall x \in H^{*}\}$. Let $\sigma \in H$. To show $\sigma \in H^{**}$, we need to show $\sigma(x)=x\ \forall x \in H^*$. But by defintion of $H^*$, for $x \in H^*$ then  $\tau(x)=x \ \forall \tau \in H $, hence in particular $\sigma(x)=x$. This is true for all $x \in H^*$ so indeed $\sigma \in H^{**}$. Since $\sigma$ was arbitrary, this shows $H\subseteq H^{**}$.
\item By definition, $K^*=$Gal$(E:K)=\{ \tau \in \text{Gal}(E:F) \ | \ \tau(x)=x \ \forall x \in K\}$. $K^{**}=\{x\in E \ | \ \tau(x)=x \ \forall \tau \in K^*\}$. Again, let $x\in K$. To show $x\in K^{**}$ we need to show $\tau(x)=x$ for all $\tau \in K^*$. But again this is true by the definition of $K^*$. Since $x$ was arbitrary this shows that $K\subseteq K^{**}$. 
\item Suppose $K\subseteq L$. Again, $L^*=\{ \tau \in \text{Gal}(E:F) \ | \ \tau(x)=x \ \forall x \in L\}$. In particular this means that for every $\tau$ in $L^*$, since $\tau$ fixes $L$, it also fixes $K\subseteq L$. Hence if $\tau \in L^*$, $\tau(x)=x \ \forall x \in K$. This means that $\tau \in K^*=\{ \tau \in \text{Gal}(E:F) \ | \ \tau(x)=x \ \forall x \in K\}$. Since $\tau$ arbitrary, this shows that $L^*\subseteq K^*$.\\
\\
Now suppose $H\subseteq I$. Again, $I^*=\{x\in E \ | \ \tau(x)=x \ \forall \tau \in I\}$. This means that for every $x$ in $I^*$, since $x$ is stabilised by every element of $I$, and elements of $H$ are also elements of $I$, $x$ is in particular stabilised by $H$. Hence if $x \in I^*$, $\tau(x)=x \ \forall \tau \in H$. This means that $x \in H^*=\{x \in E \ | \ \tau(x)=x \ \forall x \in H\}$. Since $x$ arbitrary, this shows that $I^*\subseteq H^*$.
\item We have by applying $ii)$ on $H^*\in \mathcal{F}$, that $H^*\subseteq (H^*)^{**}$. For the other direction, observe that by $i)$, $H\subseteq H^{**}$. Now applying $iii)$, this implies that $H^*\supseteq (H^{**})^*$ which is what we wanted to show. Hence $H^*=H^{***}$. We can show that $K^*=K^{***}$ in exactly the same way. 
\end{enumerate}
$\newline$
\textbf{Solution 4}
\begin{enumerate}[(a)]
\item Suppose that $K^{**}=K$ for each $K\in \mathcal{F}$. Then clearly $*:\mathcal{H}\rightarrow \mathcal{F}$ is surjective since for any $K\in \mathcal{F}$, $K^{*}\in \mathcal{H}$ and $(K^*)^*=K$. Similarly, if $*:\mathcal{H}\rightarrow \mathcal{F}$ is surjective, for any $K\in \mathcal{F}$ we can write $K=H^*$ for some $H\in \mathcal{H}$. Then $K^{**}=H^{***}=H^{*}=K$ by point $(iv)$ of Exercise 3. 
\item The argument is analogous. 
\end{enumerate}

\textbf{Solution 5.}\\
Let $L\supseteq K$ be a finite Galois extension, and let $G=$Gal$(L:K)$ be its Galois group. 
For an element $a\in L$, we denote by $orb_G(a)=\{\sigma(a) \ | \ \sigma \in G\}$ the orbit of $a$ by $G$. Also, we denote by $Stab_G(a)=\{\sigma \in G\ \ | \ \sigma(a)=a\}$ the stabilizer of $a$ in $G$. Let $p(x)= \prod_{a_i\in orb_G(a)} (x-a_i)\in L[x]$

First observe that $\sigma(p(x))=p(x)$ $\forall \sigma \in$  Gal$(L:K)$ and since $K\subseteq L$ is Galois we deduce that $p(x)\in K[x]$.

Let $\mu_{a,K}(x)\in K[x]$ be the minimal polynomial of $a$ over $K$. Observe that $\mu_{a,K}(a)=0$ $\implies$ $\sigma(\mu_{a,K}(a))=0$ $\implies$ $\mu_{a,K}(\sigma(a))=0$ $\forall$ $\sigma\in Gal(L:K)$. We have therefore $p(x)|\mu_{a,K}(x)$ and by the previous observation $p(x)=\mu_{a,K}(x)$.

Further if $Stab_G(a)$ is trivial $|orb_G(a)|=\#Gal(L:K)$ (by the Orbit-stabiliser theorem) and by the previous result $deg(\mu_{a,K})=\#Gal(L:K)$. We conclude that $deg(\mu_{a,K})=[L:K]$ (since $K\subseteq L$ Galois), which implies that $L=K[a]$.

\textbf{Solution 6.}\\
Without loss of generality we may assume that $a_1,...,a_n$ are square free integers.
\begin{itemize}
    \item The extension $\mathbb{Q}(\sqrt{a_1}, ... , \sqrt{a_n})\supseteq \mathbb{Q}$ is the splitting field of the separable polynomials $x^2-a_1$,..., $x^2-a_n$ and is therefore Galois. We have a group homomorphism $Gal(\mathbb{Q}(\sqrt{a_1}, ... , \sqrt{a_n})\supseteq \mathbb{Q})\xrightarrow{} \prod_{i=1}^n C_2$ (where $C_2$ is the group ${1, -1}$ under multiplication) which is defined by $\sigma\in Gal(L:K)\mapsto (\frac{\sigma(a_1)}{a_1}, \dots, \frac{\sigma(a_n)}{a_n})$. This map is clearly injective, it is also surjective because $\#Gal(\mathbb{Q}(\sqrt{a_1}, ... , \sqrt{a_n}):\mathbb{Q})=[\mathbb{Q}(\sqrt{a_1}, ... , \sqrt{a_n}):\mathbb{Q}]$ (since the extension is Galois) and $[\mathbb{Q}(\sqrt{a_1}, ... , \sqrt{a_n})]:\mathbb{Q}=2^n$ (by last week's exercise sheet). We have $Gal(\mathbb{Q}(\sqrt{a_1}, ... , \sqrt{a_n})\supseteq \mathbb{Q})\simeq \prod_{i=1}^n C_2$. 
    
    \item By our explicit description of the Galois group (which we will henceforth denote by $G$), $G$ is generated by $\sigma_i$ which satisfies $\sigma_i(\sqrt{a_j})=(-1)^{\delta_{ij}}\sqrt{a_j}$. Since $\sum_{i=1}^{n}\sqrt{a_i}$ is not fixed by any $\sigma_i$, it is a primitive element. (by Ex. 5).
    
    \item Consider the tower of extensions $\mathbb{Q}\subseteq \mathbb{Q}[1 + \sqrt{2} + \sqrt{3} + ... + \sqrt{N}]\subseteq \mathbb{Q}[\sqrt{2},...,\sqrt{p_k}]$ (the top field is $\mathbb{Q}$ adjoined with the square roots of the primes less than $N$). There is a $\sigma\in G$ ($G:=Gal(\mathbb{Q}[\sqrt{2},...,\sqrt{p_k}]:\mathbb{Q})$ ) s.t. $\sigma(\sqrt{2})=-\sqrt{2}$ and $\sigma(\sqrt{p})=\sqrt{p}$ for all other primes $p$. We have $\sigma(\Sigma_{k=1}^{\floor{N}}\sqrt{k})-\Sigma_{k=1}^{\floor{N}}\sqrt{k}=\Sigma_{k=1}^{\lfloor{N/2}\rfloor}\sqrt{2k}\neq 0$. Since $\mathbb{Q} \subseteq \mathbb{Q}[\sqrt{2},...,\sqrt{p_k}]$ is Galois, we have the required result.
    
    \end{itemize}
\end{document}
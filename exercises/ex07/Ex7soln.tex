\documentclass[12pt,a4paper]{article}
\usepackage{fancyhdr}
 
\pagestyle{fancy}
\fancyhf{}
\lhead{Math-317 Galois Theory \\ Problem Set 7 - Solutions}
\rhead{November 4, 2019}
\rfoot{5,6 by J.Baudin \\1,2 by S.Giampietro\\3,4 by V.A.Nadarajan}
 

\usepackage{times}
\usepackage{mathptmx}
\usepackage{mathrsfs}
\usepackage{amssymb}
\usepackage{fancyhdr}
\usepackage{amsmath, amsthm}
\usepackage{xfrac}    
\usepackage{tikz-cd}
\usepackage{enumitem}
\usepackage{commath}
\usepackage{enumerate}


\newcommand{\Q}{\mathbb{Q}}
\newcommand{\F}{\mathbb{F}}
\newcommand{\Lbar}{\overline{K}}
\newcommand{\id}{\text{id}}
\newtheorem{Lemma}{Lemma}
\newcommand{\Hom}[2]{\text{Hom}_{#1}({#2},\overline{K})}
\newcommand\restrict[1]{\raisebox{-.5ex}{$|$}_{#1}}




\begin{document}


\vskip 2ex


\textbf{Solution 1.  }\\
We first recall/prove the following important facts:
\begin{Lemma} Let $K(u)\supseteq K$ be a simple extension generated by an element $u$ and let $p(x)\in K[x]$ be its minimal polynomial and let $u=u_1, u_2, ..., u_k$ be its distinct roots. Then $|\Hom{K}{K(u)}|=k=\#\{$ distinct roots of $p\ \}$. 
\end{Lemma}
This follows from the fact that any $\phi\in \Hom{K}{K(u)}$ is uniquely determined by the image of $u$. We know that the image of $u$ must also be a root of $p$, and that for each different root $u_i$, $\phi: u\mapsto u_i$ gives rise to a different morphism in $\Hom{K}{K(u)}$, so we indeed have that $|\Hom{K}{K(u)}|=k=\#\{$ distinct roots of $p\ \}$.\\
\\
More generally you have shown in problem set 5 that:
\begin{Lemma}
Let $L\supseteq K$ be a finite field extension. $|\Hom{K}{L}|\leq [L:K]$
\end{Lemma}
\begin{Lemma}
Let $M\supseteq L\supseteq K$ be a tower of finite field extensions. Then $|\Hom{K}{M}|=|\Hom{L}{M}|\cdot |\Hom{K}{L}|$. (\textbf{Remark:} A priori $|\text{Hom}_{E}(F,\overline{E})|$ depends on a given embedding of $E$ in $\overline{E}$, the proof will show that it is the same for all embeddings)
\end{Lemma}
\begin{proof}Let $\Hom{K}{L}$ be the family of morphisms into $\overline{K}$ that extend the embedding $\iota$. For each $\sigma \in \Hom{K}{L}$, consider $\Hom{\sigma,L}{M}$ the family of morphisms $\phi: M\rightarrow \overline{K}$ such that $\phi\restrict{L}\equiv \sigma$. Therefore each $\sigma \in \Hom{K}{L}$ admits precisely $|\Hom{\sigma,L}{M}|$ extensions to $M$.\\

We now want to show that this number is independent of $\sigma$. Indeed let $\tau \in \Hom{K}{L}$ be another morphism. Our goal is to show that \\ $|\Hom{\sigma,L}{M}|=|\Hom{\tau,L}{M}|$.\\

For this, consider the morphism $\tau\sigma^{-1}:\sigma(L)\rightarrow\tau(L)$. Then we can extend this morphism to a morphism $\lambda: \overline{K}\rightarrow\overline{K}$ such that $\lambda\restrict{\sigma(L)}\equiv \tau\sigma^{-1}$. In fact $\lambda$ is an automorphism of $\overline{K}$ since it fixes $\iota(K)$ and the extension $\iota(K)\subseteq \overline{K}$ is algebraic.

Let $\phi\in \Hom{\sigma,L}{M}$. Then $\lambda\phi: M\rightarrow \overline{K} \rightarrow \overline{K}$ and $(\lambda\phi)\restrict{L}=\lambda(\phi\restrict{L})=\lambda\sigma = \tau\sigma^{-1}\sigma=\tau$. Hence $\lambda\phi\in \Hom{\tau,L}{M}$.
Conversely, for $\psi\in \Hom{\tau,L}{M}$, we can consider $\lambda^{-1}\psi:L \rightarrow \overline{K} \rightarrow \overline{K}$ and
 $(\lambda^{-1}\psi)\restrict{L}=\lambda^{-1}(\psi\restrict{L})=\lambda\tau = \sigma\tau^{-1}\tau=\sigma $ so $\lambda^{-1}\psi\in \Hom{\sigma,L}{M}$. \\
 Hence there is a bijection between the sets $\Hom{\sigma,L}{M}$ and $\Hom{\tau,L}{M}$ and this shows that $|\Hom{\sigma,L}{M}|=|\Hom{\tau,L}{M}|$.  In particular for any $\sigma$, $|\Hom{\sigma,L}{M}|=|\Hom{L}{M}|$ if we choose $\tau$ to be our fixed embedding of $L$ in $\overline{K},$ $\iota_L$. \\

Finally noting that the restriction to $L$ of each morphism in $\Hom{K}{M}$ is a morphism in $\Hom{K}{L}$,  we obtain that: 
\begin{align*}
|\Hom{K}{M}|&=\sum_{\sigma\in \Hom{K}{L}}|\Hom{\sigma,L}{M}|\\
&=\sum_{\sigma\in \Hom{K}{L}}|\Hom{L}{M}|=|\Hom{L}{M}|\cdot |\Hom{K}{L}|
\end{align*}
\end{proof}
We now prove the statement of the exercise. \\
\\
Firstly, suppose $L\supseteq K$ is separable. Then by the primitive element theorem, there exists $\alpha \in L$ such that $L=K(\alpha)$. Then the minimal polynomial $p_\alpha(x)\in K[x]$ of $\alpha$ over $K$ has degree $[L:K]=n$, and since it is a separable polynomial, all its root are distinct, in particular $p$ has $n$ roots. So by Lemma 1, we conclude that: 
$$ |\Hom{K}{L}| = \#\{\text{ distinct roots of } p\ \} = n = [L:K]$$

Conversely, suppose that $ |\Hom{K}{L}| =n = [L:K]$. Suppose by contradiction that there exists $u\in L$ such that $u$ is not separable and let $p(x)\in K[x]$ be its minimal polynomial.  Using Lemma 3 on the tower $L\supseteq K(u) \supseteq K$, we have that $|\Hom{K}{L}|=|\Hom{K(u)}{L}|\cdot |\Hom{K}{K(u)}|$. By Lemma 2, $|\Hom{K(u)}{L}|\leq[L:K(u)]$. Now since $u$ is not separable, it's minimal polynomial has multiple roots, so the number of distinct roots of $p$ is \textit{strictly} smaller than its degree hence $|\Hom{K}{K(u)}| < \textit{deg}(p) = [K(u):K]$. But this would imply: 
\begin{align*}
&n = [L:K] =  |\Hom{K}{L}|= |\Hom{K(u)}{L}|\cdot |\Hom{K}{K(u)}| \\
&< [L:K(u)]\cdot [K(u):K] = [L:K]=n
\end{align*}
Hence we must have that every element in $L$ is separable, and hence $L\supseteq K$ is a separable extension. \\
\\
\textbf{Solution 2. }\\
This follows directly from the previous exercise. If $M\supseteq L$ is separable then, by exercise 1, $|\Hom{L}{M}|=[M:L]$. Similarly $L\supseteq K$ being separable implies that $|\Hom{K}{L}|=[L:K]$. So finally, using Lemma 3: $$|\Hom{K}{M}|=|\Hom{L}{M}|\cdot |\Hom{K}{L}|= [M:L][L:K] = [M:K]$$
So $|\Hom{K}{M}|=[M:K]$ and again by exercise 1 this shows that $M\supseteq K$ is a separable extension. \\
\\
\textit{Remark:} The statement $[M:K]_{\text{sep}}=[M:L]_{\text{sep}}[L:K]_{\text{sep}}$ is just Lemma 3.
\\

\textbf{Solution 3.  }\\
We prove this statement by induction on the number of generators $n$
\begin{itemize}
    \item \textbf{n=1} 
    
      Let $K\subseteq K(a)$ be a finite extension of degree m s.t. $a$ is separable. Then given an embedding $K\hookrightarrow \overline{K}$ (i.e. an injective homomorphism) there are m extensions to $K(a)$ -- we have $K(a)=K[x]/<\mu_{a,K}(x)>$  ($\mu_{a,K}(x)$ is the minimal polynomial of $a$ over K) and every $K$-homomorphism mapping $x$ to a root of $\mu_{a,K}(x)$ in $\overline{K}$ gives an extension and $\mu_{a,K}(x)$ has $m$ distinct roots in $\overline{K}$ since it is separable. By exercise 1 of this sheet it follows that $K(a)$ is separable.
      
    \item $\mathbf{n\xrightarrow{}n+1}$ 
    
      Consider a finite extension $K\subseteq K(a_1,...,a_{n+1})$ s.t. $a_1,...,a_{n+1}$ are separable elements over $K$. Note that by the above point, $K(a_1,...,a_{n+1})$ is a separable extension of $K(a_1,...,a_n)$ since $a_{n+1}$ is separable over $K$ (a fortiori over $K(a_1,...,a_n)$). Further by induction hypothesis $K(a_1,...,a_n)$ is a separable extension of K. The result follows from exercise 2. 
\end{itemize}{}

\textbf{Solution 4. }\\

Let $K\subseteq L$ be a finite extension with char(K)=p. We denote by $L^{\text{sep}}$ the set of elements of $L$ which are separable over $K$. If $a,b\in L^{\text{sep}}$, by exercise 3 $K(a,b)\subseteq L^{sep}$. This shows that $a+b,ab$ and $a/b$ if $b\neq 0$ are in $L^{\text{sep}}$, in other words it is a subfield of $L$.

We show next that $L^{\text{sep}}\subseteq L$ is purely inseparable. Let $a\in L$, we see that the minimal polynomial of $a$, $\mu_{a,K}(x)=f(x^{p^n})$ for some $n\geq 0$ and $f\in K[x]$ s.t. not all the degrees of the monomials of $f$ are divisible by $p$. Observe that $f$ is also irreducible and is separable since $f'\neq 0$. We conclude that $a^{p^n}$ is separable over $K$ (it has minimal polynomial $f$), in other words $a^{p^n}\in L^{\text{sep}}$. 

Let $E\subseteq F$ be a finite purely inseparable extension. We want to show that $|Hom_F(E,\overline{F})|=1$. This follows easily from the fact that $\forall n\geq 0$ and $a\in F$ $X^{p^n}-a$ has a unique root in $\overline{F}$. (Since if $\alpha\in \overline{F}$ is a root then $X^{p^n}-a=X^{p^n}-\alpha^{p^n}=(X-\alpha)^{p^n}$.) So we map every element in $E$ to the unique root (in $\overline{F}$) of its minimal polynomial over $F$.

Let $K\subseteq L$ be a finite extension. We have $[L:K]_{\text{sep}}=|Hom_K(L,\overline{K})|=|Hom_K(L^{\text{sep}},\overline{K})|$ by the last two paragraphs. (the last two paragraphs show the map $Hom_K(L,\overline{K})\xrightarrow{} Hom_K(L^{\text{sep}},\overline{K})$ defined by $\phi \mapsto \phi|_{L^{\text{sep}}}$ is a bijection. Look at lemma 3 in solution 1) By exercise 1, since $K\subseteq L^{\text{sep}}$ is separable, we have $|Hom_K(L^{\text{sep}},\overline{K})|=[L^{\text{sep}}:K]$. 

\textbf{Solution 5.  }
	\begin{itemize}
		\item Note that $\sqrt{3}(\sqrt[3]{5} - \sqrt[5]{7}) \in \mathbb{Q}(\sqrt{3}, \sqrt[3]{5}, \sqrt[5]{7})$ and $\mathbb{Q}(\sqrt{3}, \sqrt[3]{5}, \sqrt[5]{7}) \supseteq \mathbb{Q}$ is a radical extension. \\ Indeed, one can take, according to the notations of definition $3.2$ in the lecture notes, $E_0 = \mathbb{Q}(\sqrt{3}, \sqrt[3]{5}, \sqrt[5]{7})$, $E_1 =  \mathbb{Q}(\sqrt{3}, \sqrt[3]{5})$, $E_2 = \mathbb{Q}(\sqrt{3})$, $E_3 = \mathbb{Q}$.
		\item Note that $(\sqrt{3} - 5)(4 - 3\sqrt[5]{6}) \in \mathbb{Q}(\sqrt{3}, \sqrt[5]{6})$ which is again a radical extension of $\mathbb{Q}$ (take $E_0 = \mathbb{Q}(\sqrt{3}, \sqrt[5]{6}), E_1 = \mathbb{Q}(\sqrt{3}), E_2 = \mathbb{Q})$.
	\end{itemize}
\textbf{Solution 6.} \\
Recall the following fact from group theory : let $p$ be a prime number, let $\tau \in S_p$ be any transposition and let $\sigma \in S_p$ be any $p$-cycle. Then $S_p$ is generated $\sigma$ and $\tau$. Recall that this follows since every transposition is obtained by conjugating $\tau$ by powers of $\sigma$. \\

Let $F$ be a splitting field of $f$ and let $G$ be the Galois group of the extension $F \supseteq \mathbb{Q}$. Then we know that $G$ acts on the roots of $f$ (i.e. the image of a root of $f$ by an element of $G$ remains a root of $f$) and that this action is faithful (because $F$ is generated by the roots of $f$ as a field by definition of a splitting field). Since $f$ has exactly $p$ roots ($f$ is irreducible and $\mathbb{Q}$ is perfect so $f$ is separable), we obtain an embedding of groups \[G \hookrightarrow  S_p\] so from now on, we will see $G$ as a subgroup of $S_p$. \\
We want to show that $G$ contains a transposition and a $p$-cycle to conclude that $G = S_p$. Since the extension $E \supseteq \mathbb{Q}$ is a splitting field of a separable polynomial, it is a Galois extension so $|G| = [F : \mathbb{Q}]$. Let $a$ be any root of $f$ and consider \[F \supseteq \mathbb{Q}(a) \supseteq \mathbb{Q}\] By the tower law, \[[F : \mathbb{Q}] = [F : \mathbb{Q}(a)][\mathbb{Q}(a) : \mathbb{Q}] = p[F : \mathbb{Q}(a)]\] so $p$ divides $[F : \mathbb{Q}] = |G|$. By Cauchy's theorem, there exists an element $\sigma \in G$ of order $p$. As $p$ is prime and $\sigma \in S_p$, $\sigma$ must be a $p$-cycle so $G$ contains a $p$-cycle. \\
Thus, we are left to show that $G$ contains a transposition. In fact, we show that the complex conjugation (denoted $\tau$) is an element of $G$ corresponding to a transposition. \\ The fact that $\tau$ fixes $\mathbb{Q}$ is immediate. Let $\alpha \in F$. We show that $\tau(\alpha) = \bar{\alpha} \in F$. Let $m(x) \in \mathbb{Q}[x]$ be the minimal polynomial of $\alpha$. Since $\tau$ is an element of the Galois group of the extension $\mathbb{C} \supseteq \mathbb{R}$, $\bar{\alpha}$ is also a root of $m(x)$. Furthermore, since $F \supseteq \mathbb{Q}$ is a normal extension, $\bar{\alpha} \in F$. Therefore, we have shown that $\tau(F) \subseteq F$. Finally, we use Ex 3 of sheet 4 to conclude that $\tau$ is an automorphism of $F$, so $\tau \in G$. \\
Let $z_0$ be a non-real root of $f$ and consider its complex conjugate $\bar{z_0} \in F$. By assumption, all the roots $\neq z_0, \bar{z_0}$ are real so they must be fixed by the complex conjugation. Thus, $\tau$ only permutes two elements so it corresponds to a transposition. Thus, $G$ contains a $p$-cycle and a transposition so it is the whole group $S_p$.


\end{document}
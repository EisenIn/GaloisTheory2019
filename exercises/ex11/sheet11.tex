\documentclass[12pt,a4paper]{article}
\usepackage{fancyhdr}
 
\pagestyle{fancy}
\fancyhf{}
\lhead{Math-317 Galois Theory \\ Problem Set 11 }
\rhead{December  3, 2019}
\rfoot{Problems selected by J. Baudin, F. Eisenbrand, S. Giampietro and V.A.Nadarajan}
\usepackage{times}
\usepackage{mathptmx}

\usepackage[utf8]{inputenc} 
\usepackage{mathrsfs}
\usepackage{mathtools} 
\usepackage{amssymb}
\usepackage{amsthm}
\usepackage{amscd}
\usepackage{hyperref}
\usepackage{enumerate}

\usepackage{../../utf8math}

\DeclareMathOperator{\gal}{Gal}

\begin{document} 
\begin{enumerate}
\item 
Recall the definition of \textit{normal closure} given in problem set 9: for $K\subseteq L$ a finite separable extension, we call the extension $L\subseteq M$ the normal closure of $L$ if $M$ is the unique minimal extension of $L$ such that $K\subseteq M$ is Galois.  \\

Let $K \subseteq E \subseteq M$, $K \subseteq F \subseteq M$ be finite field extensions. Define the compositum $EF$ as the smallest subfield of $M$ containing both $E$ and $F$. 

\begin{enumerate}[(a)]

\item Suppose $K\subseteq L$ is a radical extension. Show that for $\sigma: L\xrightarrow{}L'$ a $K$-isomorphism of fields, $K\subseteq L'=\sigma(L)$ is also a radical extension. 
\item Suppose that $K\subseteq L$ is a separable extension, and let $G=Gal(M:K)$. Then show that the normal closure of $L$, $L\subseteq M$ is the product of all conjugate fields of $L$. In other words: $M=\prod_{\sigma\in G}\sigma(L)$.
\item Show that if an extension $K\subseteq L$ is separable and radical, then its normal closure $K\subseteq M$ is also radical. \\ \textit{Hint: You may want to use the previous points and exercise 3 of review sheet 2!}
\end{enumerate}
\item We use the notation from the beginning of Chapter~4, currently page~39.

  Given the second row of $A_i$,
  explain how to compute the other rows of $A_i$ quickly with the following input:
  \begin{enumerate}[a)] 
  \item The second row of $A_i$
  \item The minimal polynomial $p(x) ∈ℚ[x]$ of $α = α_0$. 
  \end{enumerate}
 \item We use the notation from the beginning of Chapter~4, currently page~39. Describe how to fill in the group table that describes $\gal(E:ℚ)$, once the matrices $A_i$ are computed. 
\item If $α ∈ℂ \setminus \{0\}$ is a root of $p(x) = a_0 + a_1x + \cdots + a_n x^n ∈ℂ[x]$, then $1/α$ is a root of $\wt{f}(x) = a_n + a_{n-1}x + \cdots + a_0x^n$.  Also if $f(x)$ can be factored into $f(x) = p(x) q(x)$, then $\wt{f}(x) = \wt{p}(x) \wt{q}(x)$, where $\wt{p}(x)$ and $\wt{q}(x)$ are analogously defined. \label{item:20}

\item Let $α = (a + ib)∈ℂ$ be a complex number with $|α|<1$, the binary expansion of $a$ and $b$ being $a = ∑_{i=1}^∞ a_i 2^{-i}$ and  $b = ∑_{i=1}^∞ b_i 2^{-i}$, where the $a_i, b_i ∈ \{0,1\}$. The numbers $a_1,\dots,a_k$ and $b_1,\dots,b_k$ are called the first $k$ bits of the real and imaginary part of $α$ respectively. In the following, let $\br{α} ∈ℂ$ be a complex number of absolute value at most one. 
  \begin{enumerate}[a)]
  \item If the first $k$ bits of the real and imaginary parts of $α$ and $\br{α}$ coincide, then $|α - \br{α}| ≤2^{k-1}$.
  \item If the first $k+1$ bits of the real and imaginary parts of $α$ and $\br{α}$ coincide, then $|α^2 - \br{α}^2| ≤2^{k-2}$.
  \item If the first $k+1$ bits of the real and imaginary parts of $α$ and $\br{α}$ coincide, then $|α^j - \br{α}^j| ≤2^{k-1} j$.
  \item Conclude that, $\br{α}_i$ in Theorem~\ref{thr:35} can be set to $\br{α}_1^i$, if the real and imaginary parts of $\br{α_1}$ coincide with those of $α$ in the first $c' (n^3 + n^2 \log \|p\|_∞)$ bits, where $c'$ is an absolute constant, depending only on $c$. 
  \end{enumerate}
\end{enumerate}

\end{document}
%%% Local Variables:
%%% mode: latex
%%% TeX-master: t
%%% End:

\documentclass[12pt,a4paper]{article}
\usepackage{fancyhdr}
 
\pagestyle{fancy}
\fancyhf{}
\lhead{Math-317 Galois Theory \\ Problem Set 1}
\rhead{September 17, 2019}
\rfoot{Problems selected by F. Eisenbrand}
 

\usepackage{times}
\usepackage{mathptmx}
\usepackage{mathrsfs}
\usepackage{amssymb}
\usepackage{amsmath}

\usepackage{../../utf8math}





\begin{document}


\vskip 2ex



\begin{enumerate}
\item Let $θ: R_1 → R_2$ be a ring homomorphism. Show that $\ker(θ) = \{ r ∈ R_1 : θ(r) = 0 \}$ is an ideal of $R_1$. 
\item Let $R$ be an integral domain and consider the construction of the field  of quotients as described above. We denote the equivalence class of $(r,u)$ with $u ≠0$ by $\frac{r}{u}$. Show that addition and multiplication
  \begin{displaymath}
    \frac{r}{u} + \frac{x}{y} = \frac{ry + ux}{uy} \quad  \text{ and } \quad  \frac{r}{u} ⋅ \frac{x}{y} = \frac{rx}{uy} 
  \end{displaymath}
  where $r,u,x,y ∈R$ and $uy ≠0$ are well defined. 
\item Complete the proof of Theorem~1.1 by showing that the multiplication defined on $R/A$ is associative and that the distributive laws hold. 
 \item Let $f(x) ∈ R[x]\setminus \{0\}$ and $a ∈R$ be a root of $f(x)$ with multiplicity $m$. Let $g(x) ∈ R[x]$ with $f(x) = (x-a)^m g(x)$. Show that $a$ is not a root of $g(x)$.
\item Let $R$ be an integral domain and  $f(x) ∈ R[x]$ be a nonzero polynomial. Show that $f(x)$ has at most $\deg f(x)$ many different roots.
\item Describe an irreducible (in $ℚ[x]$) polynomial $p(x) ∈ℤ[X]$ with degree at least two which does not satisfy the Eisenstein criterion.
\item Let  $f(x),g(x) ∈ℂ[x]$ be nonzero. Show that $M(f(x)g(x)) = M(f(x)) \, M (g(x))$.
\item Let $f(x),g(x) ∈ℂ[x]$ be nonzero polynomials with leading coefficients $a,b ≠0$ where $\deg (f) = n$. Show the following generalization of Theorem~1.11: $\|g\|_1 ≤ 2^n |b| / |a| \|f\|_2$.  

  
\end{enumerate}
\end{document}






%%% Local Variables:
%%% mode: latex
%%% TeX-master: t
%%% End:

\documentclass[12pt,a4paper]{article}
\usepackage{fancyhdr}
 
\pagestyle{fancy}
\fancyhf{}
\lhead{Math-317 Galois Theory \\ Problem Set 8 suggestions  }
\rhead{November 5, 2019}
\rfoot{Problems selected by J. Baudin, F. Eisenbrand S. Giampietro and V.A.Nadarajan}
\usepackage{times}
\usepackage{mathptmx}

\usepackage[utf8]{inputenc} 
\usepackage{mathrsfs}
\usepackage{mathtools} 
\usepackage{amssymb}
\usepackage{amsthm}
\usepackage{amscd}

\usepackage{enumerate}



\begin{document} 
\noindent

\textbf{Exercise 1}\\
Let $K\subseteq L$ be a finite field extension. Denote by Hom$_K(L, \overline{K})$ the set of homomorphisms $L\rightarrow \overline{K}$ that extend a given embedding of $K$ into $\overline{K}$. Show that $K\subseteq L$ is normal iff for $\phi_1,\phi_2\in$ Hom$_K(L, \overline{K})$, $\phi_1(L)=\phi_2(L)$. Using this, give a simple proof of the fact: $K\subseteq L$ is normal if it is generated by elements $\alpha_1,...,\alpha_n$ whose minimal polynomials $\mu_{\alpha_1,K},...,\mu_{\alpha_n,K}$ split in $L[x]$. \\


\textbf{Exercise 2}\\
Let $K\subseteq L\subseteq M$ be a finite field extension. 

\begin{itemize}
    \item If $K\subseteq L$ is a normal subextension of $K\subseteq M$, then every $K$-automorphism of $M$ induces a $K$-automorphism of $L$. (i.e. if $\sigma\in Gal(M:K)$ then $\sigma|_L(L)\subseteq L$).
    \item Now assume that $K\subseteq M$ is normal and L is an arbitrary subfield, then every $K$-algebra homomorphism $\phi:L\xrightarrow{}M$ lifts to an $K$-automorphism $\widetilde{\phi}:M\xrightarrow{}M$ i.e. $\widetilde{\phi}|_L=\phi$. [Hint: Use exercise 1]
\end{itemize}{}

\textbf{Exercise 3}\\
Let $M\supseteq K$ be a finite Galois extension, and let $G=$Gal$(M:K)$ be its Galois group. 
\begin{enumerate}
    \item Let $L$ be an intermediate field in the extension $M\supseteq K$. Show that $\forall$ $\sigma\in G$ $\sigma(L)$ is also an intermediate field and $(\sigma(L))^*=\sigma L^* \sigma^{-1}$.  
    \item Let $L$ be an intermediate field in the extension $M\supseteq K$. Show that $L$ is normal over $K$ iff $\sigma(L)=L$ $\forall$ $\sigma\in G$. Using the previous result show that $L$ is normal over $K$ iff $L^*$ is a normal subgroup of $G$.
    \item Let $L$ be an intermediate field in the extension $M\supseteq K$ such that $L$ is normal over $K$. Show that $Gal(L:K)\simeq G/L^*$.
    [Hint: Consider the map $G\xrightarrow{}Gal(L:K)$ defined by $\sigma\in G\mapsto \sigma|_{L}$.]
    
\end{enumerate}{}

\textbf{Remark:} The first statement implies that conjugate subgroups give rise to $K$-isomorphic field extensions under the Galois correspondence, so to understand the lattice of intermediate fields it is sufficient to consider subgroups up to conjugation.\\

\textbf{Exercise 4}\\
 Let $f\in F[x]$, where $F$ is a field and char$(F)\neq 2$. Assume that $f$ is of degree $n$ and that its roots $u_1, ... , u_n$ in a splitting field $E\supseteq F$ are distinct. Let $G=\text{Gal}(E:F)$ viewed as a group of permutations in $S_X$, where $X=\{u_1, ..., u_n\}$ is the set of roots. 
 Define $\Delta\in E$ by $\Delta:=\prod_{i<j}(u_i-u_j)$ (for example if $n=3$, $\Delta=(u_1-u_2)(u_1-u_3)(u_2-u_3)$). We call the element $\Delta^2$ the \textit{discriminant} of $f$. 
 \begin{enumerate}[a)]
 \item Show that $\Delta^2 \in F$.
 \item Show that the permutation $\sigma \in S_X$ is even if and only if $\sigma(\Delta)=\Delta$ and that $\sigma$ is odd if and only if $\sigma(\Delta)=-\Delta$. (Recall that the parity of a permutation $\sigma$ is defined as the parity of the number of inversions for $\sigma$).
 \item Show that $G$ consists of even permutations if and only if $\Delta \in F$. 
 \item If $f=x^2+bx+c$, show that $\Delta^2=b^2-4c$, the usual discriminant. 
 \item If $f=x^3+bx+c$, show that $\Delta^2=-4b^3-27c^2$, the usual discriminant. [\textit{Hint: } $u_1+u_2+u_3 =0$ \textit{ and } $u_1u_2+u_1u_3+u_2u_3=b$ \textit{ imply that }$(u_i-u_j)^2= -b-3u_iu_j$.]
 \end{enumerate}
 $\newline$
 \textbf{Exercise 5}\\
 Show that no root of  $p(x) = x^5 - 6x -2$ lies in a radical extension of $\mathbb{Q}$. [\textit{Hint: You may use the fact that the roots of any  real polynomial of degree at most 4 are radical expressions involving the coefficients of the polynomial.}]

  $\newline$
  \textbf{Exercise 6}\\
  \begin{enumerate}[a)]
  \item 
 Show that
 \begin{displaymath}
   G = \left\{
     \begin{pmatrix}
       1 & a & b \\
       0 & 1 & c \\
       0 & 0 & 1
     \end{pmatrix} \colon a,b,c \in F \right\}
\end{displaymath}
is a solvable group for any field $F$.
\item
Show that
 \begin{displaymath}
   G = \left\{
     \begin{pmatrix}
       x & a & b \\
       0 & y & c \\
       0 & 0 & z
     \end{pmatrix} \colon x,y,z,a,b,c \in F; \, xyz \neq 0 \right\}
\end{displaymath}
is a solvable group for any field $F$.
\end{enumerate}
\end{document}
%%% Local Variables:
%%% mode: latex
%%% TeX-master: t
%%% End:

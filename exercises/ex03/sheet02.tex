\documentclass[12pt,a4paper]{article}
\usepackage{fancyhdr}
 
\pagestyle{fancy}
\fancyhf{}
\lhead{Math-317 Galois Theory \\ Problem Set 2 }
\rhead{September 24, 2019}
\rfoot{Problems selected by F. Eisenbrand and V. Nadarajan}
 

\usepackage{times}
\usepackage{mathptmx}
\usepackage{mathrsfs}
\usepackage{amssymb}
\usepackage{amsmath}

\usepackage{../../utf8math}




\begin{document}


\vskip 2ex



\begin{enumerate}
\item Compute the greatest common divisor of the rational polynomials
  \begin{eqnarray*}
   f_0(x) & = &  x^{5} - 3 x^{4} + x^{3} + 8 x^{2} - 12 x + 6 \\
f_1(x) & = & x^{4} - x^{3} - x^{2} + 6
  \end{eqnarray*}
  and compute polynomials $g_0(x), g_1(x) ∈ ℚ[x]$ with 
  \begin{displaymath}
    g_0 f_0 + g_1 f_1 = \gcd(f_0,f_1). 
  \end{displaymath}

 \emph{Hint: Check the sage code that I wrote to latex the example-run of the Euclidean algorithm and modify it. The code can be found in the directory called {\tt Programming} in the GitHub repository of our course.} 
\item Which of the following polynomials in   $\mathbb{Q}[x]$ are separable?  
  \begin{eqnarray*}
    f_1(x) & = & (x^4 + 15 x^2 + 5) (x^{20} + 3x^{15} + 3 ) \\
    f_2(x) & = & x^{5} - 6 x^{4} + 7 x^{3} + 14 x^{2} - 30 x + 18      
  \end{eqnarray*}

 \item Consider the polynomial $f(x)  =   x^{11}+ 8 x^9 - 2 x^{4} + 10 x^{3} + 8 x^{2} - 12 x + 6$ and let $u ∈ℂ$ be a root of $f(x)$. Express $1/u$ as a rational linear combination of $1,u,\dots,u^{10}$. 
\item Show that the quotient ring $\mathbb{Q}[X]/(X^2-1)$ is isomorphic to the product ring $\mathbb{Q}\times \mathbb{Q}$.

  \emph{Hint: Use the Chinese remainder theorem}   
\item Let $E ⊇ F$ be a field extension of finite degree. Show that each element of $E$ is algebraic over $F$.    

\item Let $\mathbb{Z}[i]=\{a+bi|a,b\in \mathbb{Z}\}$ be the ring of Gaussian integers. (it is viewed as a ring by considering it as a subring of $\mathbb{C}$) Let $p$ be any prime such that $p\equiv 3\mod 4$. Show that the quotient ring $\mathbb{Z}[i]/〈p〉 \simeq \mathbb{F}_p[X]/〈X^2+1〉 \simeq \mathbb{F}_{p^2}$ where $\mathbb{F}_{p^2}$ is the unique (up to isomorphism) field with $p^2$ elements.
\item Let $p$ be a prime number. Show that there is a unique (up to isomorphism) field with $p^n$ elements for every natural number $n$.

  \emph{Hint: This field is the splitting field of $X^{p^n}-X$ over $F_p$.} 
\item Let $R$ be an integral domain and $Q(R)$ its field of fractions. Let $i:R\xrightarrow{}Q(R)$ denote the cananical inclusion $r\mapsto \frac{r}{1}$.  Let $\alpha:R\xrightarrow{}K$ be an injection into a field $K$. Show that there is an injection $\alpha':Q(R)\xrightarrow{}K$ s.t. $\alpha=\alpha'\circ i$. In this sense the field of fractions is the "smallest field containing R".
\item Let $R$ be an integral domain. Show that $R[X]$ is an integral domain.

  \emph{  Hint: If $f,g\in R[X]$ show that $deg(fg)=deg(f).deg(g)$.}

\end{enumerate}
\end{document}

%%% Local Variables:
%%% mode: latex
%%% TeX-master: t
%%% End:

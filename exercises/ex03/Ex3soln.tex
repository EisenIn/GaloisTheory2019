\documentclass[12pt,a4paper]{article}
\usepackage{fancyhdr}
 
\pagestyle{fancy}
\fancyhf{}
\lhead{Math-317 Galois Theory \\ Problem Set 3 - Solutions}
\rhead{October 2, 2019}
\rfoot{1,4 by J.Baudin \\2,7,8 by S.Giampietro\\3,5,6 by V.A.Nadarajan}
 

\usepackage{times}
\usepackage{mathptmx}
\usepackage{mathrsfs}
\usepackage{amssymb}
\usepackage{amsmath}






\begin{document}


\vskip 2ex


\textbf{Solution 1.}

\begin{enumerate}
\item Let $f(x) = \sum_{i = 0}^na_ix^i$ and $g(x) = \sum_{j = 0}^mb_jx^j$ and let $c = content(f)$, $d = content(g)$. We may write $a_i = ca_i'$ and $b_j = db_j'$ with $a_i',b_j'\in R$. 

Now define $f'(x) = \sum_{i = 0}^na_i'x^i$ and $g'(x) = \sum_{j = 0}^mb_j'x^j$. We then see that both $f'$ and $g'$ are in $R[x]$ and are primitive (i.e. their content is a unit). Suppose that we have shown that $f'g'$ is also primitive, we would conclude that \[ content(fg) = content(cdf'g') = cd*content(f'g') = cd \] so we would be done.  \\ Hence, we are left to show the following lemma : let $f, g \in R[x]$ be two primitive polynomials, then their product $fg$ is also primitive. 

Suppose the contrary and let $\pi \in R$ be an irreducible divisor of each term of $fg$. Since $R$ is a UFD (unique factorization domain), we deduce that $\pi$ is prime. Therefore, $F = R/\langle \pi \rangle$ is an integral domain (so $F[x]$ is an integral domain too). Let $\overline{f}, \overline{g}$ denote the images of $f, g$ through the quotient by $\langle \pi \rangle$. Since $\pi$ divides $fg$ we obtain that \[ 0 = \overline{fg} = \overline{f}\overline{g} \] Thus, either $\overline{f}$ or $\overline{g}$ is $0$ in the quotient so equivalently, $\pi$ divides either $f$ or $g$. However, this contradicts the fact that both $f$ and $g$ were primitive.

\item Write $f(x) = \sum_{j = 0}^na_jx^j$ (with $a_n = 1$) and $a = b/c$ where $b, c \in R$ are coprime (we can assume that they are coprime because we work in a UFD). Then, since $a$ is a root of $f$, we obtain that \[ 0 = a_0 + a_1\frac{b}{c} + \dots + a_{n- 1} \left(\frac{b}{c}\right)^{n - 1} + \left(\frac{b}{c}\right)^n \] Multiplying by $c^n$ both sides, we have \[ 0 = c^na_0 + c^{n - 1}ba_1  + \dots + a_{n - 1}cb^{n - 1} + b^n \] so \[ -b^n = c(c^{n - 1}a_0 + c^{n - 2}ba_1  + \dots + a_{n - 1}b^{n - 1}) \] Hence, $c$ divides $-b^n$. However, recall that $b$ and $c$ are coprime, so the only way that $c$ divides $-b^n$ is that $c$ has no irreducible divisor, i.e. $c \in R^\times$. Thus, $a = \frac{b}{c} \in R$.

\item Before actually showing this, we show the following lemma : \\ Let $f, g \in R[x]$ be primitive polynomials such that there exists $a \in K$ with $f(x) = ag(x)$. Then $a \in R^\times$. 

Let $a = \frac{b}{c}$, then $cf(x) = bg(x)$. Therefore, $content(cf(x)) = content(bg(x))$. Since $f$ and $g$ are primitive, we obtain that $c = b$ up to a unit (because the content is defined up to a unit). Hence, $a = \frac{b}{c} \in R^\times$. 

Now we go back to our problem. Let $g(x) = \frac{a_0}{b_0} + \dots + \frac{a_n}{b_n}x^n$ and $h(x) = \frac{a_0'}{b_0'} + \dots + \frac{a_m'}{b_m'}x^m$. Define $d_1 = gcd(a_0, \dots, a_n)$, $m_1 = lcm(b_0, \dots, b_n)$ $d_2 = gcd(a_0', \dots, a_m')$ and $m_2 = lcm(b_0', \dots, b_m')$. We then obtain that the polynomials $p(x) = \frac{m_1}{d_1}g(x)$ and $q(x) = \frac{m_2}{d_2}h(x)$ have coefficients in $R$ and are actually primitive. Moreover, we have \[ f(x) = \frac{d_1d_2}{m_1m_2}p(x)q(x) \] By the little lemma we have proven before, the term $\frac{d_1d_2}{m_1m_2}$ is a unit of $R$ so we are done.

\item First suppose that $f$ is irreducible in $K[x]$ and let $g \in R[x]$ be a divisor of $f$ in $R[x]$. Then, $g$ is also a divisor of $f$ in $K[x]$ so $g$ must be a unit of $K[x]$, i.e. $g$ is a constant. However, since $f$ is primitive, $g$ is in fact a unit. Hence, we have proven that $f$ is irreducible in $R[x]$. \\
Now suppose that $f$ is irreducible in $R[x]$ (in particular, $f$ is primitive). Suppose that there exist $g. h \in K[x]$ such that $f = gh$. Then, by the previous point, there exist $p, q \in R[x]$ primitive polynomials such that $f = pq$ and $deg(p) = deg(g)$, $deg(q) = deg(h)$ (cf the proof of the previous point). Since $f$ is irreducible in $R[x]$, either $p$ or $q$ is a unit. Hence, either $g$ or $h$ is a constant (so a unit in $K[x]$). Therefore, $f$ is indeed irreducible in $K[x]$.

\item Let $f \in R[x]$ and suppose for the moment that $f$ is primitive. We show that there exists a unique decomposition of $f$ as a product of irreducibles in $R[x]$. First, since $K$ is a field, $K[x]$ is a euclidean domain so it is in particular a UFD. Thus, there exist irreducibles $g_1, \dots, g_n \in K[x]$ such that \[f = g_1 \cdots g_n\] Then, using Gauss' Lemma inductively, there exist $f_1, \dots, f_n \in R[x]$ primitive polynomials such that \[f = f_1 \cdots f_n \] and $f_i = a_ig_i$ for some $a_i \in K$. Since the $g_i$ are irreducibles in $K[x]$, the polynomials $f_i$ are also irreducible in $K[x]$ and since they are primitive, by the previous point, they are irreducible in $R[x]$ so we have found a decomposition into irreducibles of $f$. \\ We show that it is unique. Let $f = f_1' \cdots f_m'$ be a decomposition into irreducibles of $R[x]$. By the previous point, all these $f_i$'s are also irreducible in $K[x]$ so, by the uniqueness of the factorization in $K[x]$, $n = m$ and there exist $b_1, \dots, b_n \in K$ such that, up to reordering the indexes, $f_i' = b_ig_i'$ for all $i$. Then, for all $i$, $f_i' = \frac{b_i}{a_i}f_i$. Therefore, by the little lemma we proved in the beginning of $(c)$, $\frac{b_i}{a_i} \in R^\times$ for all $i$. Thus, the decomposition is indeed unique in $R[x]$ for primitive polynomials. \\
Now let $f \in R[x]$ be any polynomial and let $c = content(f)$. Define $f' = \frac{1}{c}f$ ($f'$ is then primitive). We obtain that there exist a unique factorization for $f'$. Since $R$ is a UFD, there also exists a unique factorization for $c$ and since $f = cf'$, there exists a factorization for $f$ into irreducibles. We now show the uniqueness of this factorization. Let $f = a_1 \cdots a_nf_1 \cdots f_m = a_1' \cdots a_k' f_1' \cdots f_l'$ be two factorizations where all the $a_i, a_j'$ are irreducible constants and all the $f_i, f_j'$ are irreducible polynomials of degree $> 0$ (so they are all primitive in particular). Then, using $(a)$ inductively, we obtain that $c = content(f) = a_1 \cdots a_n = a_1' \cdots a_k'$. Since $R$ is a UFD, this decomposition is unique so, up to a reordering and units, $k = n$ and $a_i = a_i'$ for all $i$. Then, up to a unit, $cf_1 \dots f_m = cf_1' \cdots f_l'$ so since $R[x]$ is an integral domain, $f_1 \cdots f_m = f_1' \cdots f_l'$ up to a unit. Therefore, we can apply what we have proven before for primitive polynomials : $m = l$ and, up to reordering and units, $f_i = f_i'$ for all $i$. We have proven that out two decompositions into irreducibles were the same up to units, which concludes the proof.
\end{enumerate}

\textbf{Solution 2.  }\\

We assume known that the mapping from $F[x]\rightarrow \bar{F}[x]$ with $f\mapsto f^{\pi}$ is an isomorphism of rings. 
We consider the map from $F[x]/\langle p(x)\rangle \rightarrow \bar{F}[x]/\langle p^{\pi}(x)\rangle$ given by: 
$$ \Psi_3(h(x)+\langle p(x)\rangle)=h^{\pi}(x)+\langle p^{\pi}(x)\rangle $$
\begin{itemize}
\item We first show it is well defined. We need to show that for $h(x), g(x)\in F[x]$, if $h(x)- g(x)\in \langle p(x)\rangle$, then $h^{\pi}(x)-g^{\pi}(x) \in \langle p^{\pi}(x)\rangle $. Suppose $h(x)- g(x)= p(x)q(x)$ for a certain $q(x)\in F[x]$.
Since we know the map $f\mapsto f^{\pi}$ is an isomorphism of rings, we have: 
$$
h^{\pi}(x)-g^{\pi}(x)=(h(x)- g(x))^{\pi}=(p(x)q(x))^{\pi}=p^{\pi}(x)q^{\pi}(x) \in \langle  p^{\pi}(x)\rangle 
$$
So the application is well defined. 
\item Once we have shown that it is well defined, the fact that it is an isomorphism of fields follows directly from the fact that the map $f\mapsto f^{\pi}$ is an isomorphism of rings:
\begin{align*}
\Psi_3((h(x)\langle p(x)\rangle)+(g(x)\langle p(x)\rangle))&=\Psi_3((h(x)+g(x))+\langle p(x)\rangle)\\
&=h^{\pi}(x)+g^{\pi}(x)+\langle p^{\pi}(x)\rangle \\
&= \Psi_3(h(x)+\langle p(x)\rangle)+\Psi_3(g(x)+\langle p(x)\rangle)
 \end{align*}
Similarly for multiplication. 


\end{itemize}


 
 \textbf{Solution 3. }

\begin{itemize}
    \item $11\equiv 3 (mod 4)$. By solution to exercise 6 in Sheet 2, $x^2+1$ is irreducible in $\mathbb{F}_{11}[x]$.
    \item Let K be a field and $p(x)=ax^2+bx+c$ with $a\neq 0$. We have $p(x)=a(x+\frac{b}{2a})^2-(-c+\frac{b^2}{4a})$, from that we see that $p(X)$ is irreducible (which is equivalent to $p(x)$ does not split) iff $b^2-4ac$ (the discriminant of $p$) is not a square in $K$. We compute the discriminant of the polynomial $X^2+X+4$ to be $-15\equiv -4 (mod 11)$ which is not a square since $x^2+1$ is irreducible in $\mathbb{F}_{11}[x]$.
    \item Now let $K=\mathbb{F}_{11}[y]/<y^2+y+4>$. (which is a field by the previous part) We define a ring homomorphism $\phi: \mathbb{F}_{11}[x] \xrightarrow{} K$ which maps $x$ to the residue class of $y-5$. We see that $(y-5)^2+1=y^2-10y+26=y^2+y+4 (mod 11)$ but the latter is $0$ in $K$, it follows that $ker(\phi)=<x^2+1> $. By first isomorphism theorem this induce an injection $\mathbb{F}_{11}[x]/<x^2+1>\hookrightarrow K$, this is in fact an isomorphism since $\phi(X+5)=y$.
\end{itemize}{}

\textbf{Solution 4.}\\

First suppose that $f$ and $g$ are coprime in $L[x]$. Then, it is clear that they remain coprime in $K[x]$ (any divisor in $K[x]$ is also a divisor in $L[x]$). \\ Now suppose that $f$ and $g$ are coprime in $K[x]$. Then there exist $a, b \in K[x]$ such that \[af + gb = 1\] Since this equality also holds in $L[x]$, $f$ and $g$ are coprime in $L[x]$. Indeed, let $d$ be any common divisor of $f$ and $g$ in $L[x]$. Then $d$ also divides $af + bg = 1$ so $d$ is a constant (i.e. a unit of $L[x]$). In fact the above argument can be used to show if $f$ and $g$ have g.c.d. $h\in L[x]$ then in fact $h\in K[x]$.\\



\textbf{Solution 5. }\\

Let $K$ be a field and $f\in K[x]$, $f\neq 0$ be any polynomial. Let $R=K[x]/<f(x)>$ and $\phi:K[x]\twoheadrightarrow R$ be the canonical quotient map.

Let $J\subseteq R$ be an ideal. We know that $\phi^{-1}(J)\in K[x]$ is an ideal and moreover $f\in \phi^{-1}(J)$. Since $K[x]$ is a P.I.D. $\exists$ $p(x)\in K[x]$ s.t. $\phi^{-1}(J)=<p(x)>$. (Note that by our previous observation $p|f$) Since $\phi$ is a surjection, $J=\phi(\phi^{-1}(J))=<\phi(p(x))>$. So any ideal in $R$ is of the form: $J=<\phi(p(x))>$ for some $p|f$. \\

\textbf{Solution 6. }

\begin{itemize}
    \item Recall that $\zeta_n\in \mathbb{\Bar{Q}}$ is called a primitive n-th root of unity if $\{1, \zeta_n, ..., \zeta_n^{n-1}$ is the set of roots of $x^n-1$. Note that a primitive root of unity exists (not unique in general) for every $n\geq 1$, since the roots of $x^n-1$  form a finite subgroup of the field $\mathbb{\Bar{Q}}$ and so is cyclic.
    
    Let $\zeta_3\in\mathbb{\Bar{Q}}$ be the primitive 3 rd root of unity. We leave it to the reader to check $-\zeta_3$ is a primitive 6 th root of unity. (check that ($(-\zeta_3)^k\neq 1$ $\forall$ $k|6$). Let $\zeta_6=-\zeta_3$, we have $\mathbb{Q}[\zeta_6]=\mathbb{Q}[\zeta_3]=\mathbb{Q}[x]/<x^2+x+1>$.(check!) So $[\mathbb{Q}[\zeta_6]:\mathbb{Q}]=2$.
    
    \item Let $m,n$ be positive integers. Let $l$ be the least common multiple of $m,n$ and $d$ be their g.c.d. We will show that $\mathbb{Q}[\zeta_m,\zeta_n]=\mathbb{Q}[\zeta_l]$. Since $\zeta_l^{\frac{n}{d}}$ and $\zeta_l^{\frac{m}{d}}$ are primitive mth root and nth root of unity respectively (check!),  $\mathbb{Q}[\zeta_m,\zeta_n]\subseteq \mathbb{Q}[\zeta_l]$. 
    
          Assume first that $(m,n)=1$, then $l=mn$. Let $a,b\in \mathbb{Z}$ s.t. $an+bm=1$. We will show that $\zeta_m^a\zeta_n^b$ is a primitive $mn$ th root of unity. This will show that $\zeta_l$ is a power of $\zeta_m^a\zeta_n^b$ and $\mathbb{Q}[\zeta_l]\subseteq \mathbb{Q}[\zeta_m,\zeta_n]$. Clearly $(\zeta_m^a\zeta_n^b)^l=1$ and assume that $(\zeta_m^a\zeta_n^b)^k=1$. We have from our assumption $\zeta_m^{akn}=1$ and $\zeta_n^{bkm}=1$, we get $m|akn\implies l|akn$ and similarly $n|bkm\implies l|bkm$. Therefore we deduce $l|(akn+bkm)\implies l|k$.
          
          Then we proceed by induction on number of prime factors of $d=(m,n)$. Let $m=m'p^{r_1}$ and $n=n'p^{r_2}$ s.t. $(m,p)=1$ and $(n,p)=1$ and we may assume without loss of generality that $r_1\geq r_2$. We have $l=l'p^{r_1}$ with $(l',p)=1$. By our proof of the co prime case, we get $\mathbb{Q}[\zeta_l]= \mathbb{Q}[\zeta_{l'},\zeta_{p^{r_1}}]$, and $\mathbb{Q}[\zeta_m,\zeta_n]=\mathbb{Q}[\zeta_{m'},\zeta_{n'}, \zeta_{p^{r_1}}]$ (for the second equality we use $\mathbb{Q}[\zeta_{p^{r_2}}]\subseteq \mathbb{Q}[\zeta_{p^{r_1}}]$ since $\zeta_{p^{r_1}}^{p^{r_1-r_2}}$ is a primitive $p^{r_2}$ root of unity). Since $\mathbb{Q}[\zeta_{l'}]\subseteq \mathbb{Q}[\zeta_{m'},\zeta_{n'}]$ by induction hypothesis, our result follows. 
\end{itemize}{}

\textbf{Solution 7.}

\begin{itemize}
\item We search for a primitive element for the extension $\mathbb{Q}(\sqrt{p},\sqrt{q})\supseteq \mathbb{Q}$. By the proof of the primitive element theorem, we know it is of the shape $\sqrt{p}+y\sqrt{q}$. To choose a `good' $y$, we proceed as in the notes. The minimal polynomials of $\sqrt{p}$ and $\sqrt{q}$ are respectively:
\begin{align*}
&f_p(x)=x^2-p=(x-\sqrt{p})(x+\sqrt{p})\\
&f_q(x)=x^2-q=(x-\sqrt{q})(x+\sqrt{q})
\end{align*}
Hence by the proof we can choose any $y$ from the set $\mathbb{Q} \setminus \{-\tfrac{\sqrt{p}}{\sqrt{q}}, 0\}$. 
For example with $y=1$, $\mathbb{Q}(\sqrt{p},\sqrt{q})=\mathbb{Q}(\sqrt{p}+\sqrt{q})$. \\
\\
\textit{Remark} We could have shown this last statement without the proof of the primitive element theorem. Indeed we clearly have 
$\mathbb{Q}(\sqrt{p}+\sqrt{q})\subseteq \mathbb{Q}(\sqrt{p},\sqrt{q})$. For the other direction, observe that $(\sqrt{p}+\sqrt{q})^{-1}\in \mathbb{Q}(\sqrt{p}+\sqrt{q})$ and:
$$\frac{1}{\sqrt{p}+\sqrt{q}}=\frac{\sqrt{p}-\sqrt{q}}{p-q}$$ Hence $\sqrt{p}-\sqrt{q}\in \mathbb{Q}(\sqrt{p}+\sqrt{q})$. This implies $\sqrt{p}=\tfrac{1}{2}(\sqrt{p}+\sqrt{q})+(\sqrt{p}-\sqrt{q}) \in \mathbb{Q}(\sqrt{p}+\sqrt{q})$ and similarly for $\sqrt{q}$. 

\item By the previous point, we have that $\mathbb{Q}(\sqrt{p},\sqrt{q},\sqrt{r})= \mathbb{Q}(\sqrt{p}+\sqrt{q}, \sqrt{r})$. We again search for a primitive element of the shape $(\sqrt{p}+\sqrt{q})+y\sqrt{r}$. We first need to find the minimal polynomial of $\sqrt{p}+\sqrt{q}$ over $\mathbb{Q}$. 
We verify that  $\sqrt{p}+\sqrt{q}$ is a root of the polynomial: 
\begin{align*}
f_{pq}(x)&=x^4-2x^2(p+q)+(p-q)^2\\
&=(x-\sqrt{p}-\sqrt{q})(x+\sqrt{p}-\sqrt{q})(x-\sqrt{p}+\sqrt{q})(x+\sqrt{p}+\sqrt{q})
\end{align*}
We can verify this polynomial is irreducible over $\mathbb{Q}$ by observing that none of its roots are in $\mathbb{Q}$ and that none of its quadratic factors has rational coefficients. \\
Again, the minimal polynomial of $\sqrt{r}$ over $\mathbb{Q}$ is: 
\begin{align*}
f_r(x)&=x^2-r=(x-\sqrt{r})(x+\sqrt{r})
\end{align*}
Hence can choose any $y$ from the set $\mathbb{Q} \setminus \{0, -\tfrac{\sqrt{p}}{\sqrt{r}}, -\tfrac{\sqrt{q}}{\sqrt{r}}, -\tfrac{\sqrt{p}+\sqrt{q}}{\sqrt{r}}\}$. 
For example with $y=1$, $\mathbb{Q}(\sqrt{p},\sqrt{q}, \sqrt{r})=\mathbb{Q}(\sqrt{p}+\sqrt{q}+\sqrt{r})$. \\
\end{itemize}



\textbf{Solution 8.  }\\

Observe first of all that $K$ is the fraction field of $\mathbb{F}_p[x^p, y^p]$ (the polynomial ring in the variables $x^p$ and $y^p$). Similarly, $L$ is the fraction field of $\mathbb{F}_p[x, y]$. 

\begin{enumerate}
\item Observe that $L=K(x,y)$, hence $[L:K]=[K(x,y):K(x)] [K(x):K]$. 

To compute $[K(x):K]$, we note that the minimal polynomial of $x$ over $K$ is $q_x(t)=t^p-x^p\in K[x]$. To see that this polynomial  is irreducible, by exercise 1, it suffices to show it is irreducible in $\mathbb{F}_p[x^p, y^p][t]$. This is true by Eisenstein's criterion (as $x^p$ is prime in $\mathbb{F}_p[x^p, y^p]$). Hence $[K(x):K]=p$. 

Similarly the minimal polynomial of $y$ over $K(x)$ is $q_y(t)=t^p-y^p$. Observe that $K(x)$ is the fraction field of the polynomial ring $\mathbb{F}_p[x,y^p]$ so again by exercise 1, it suffices to show that $q_y(t)$ is irreducible over $\mathbb{F}[x,y^p]$. We can again apply Eisenstein's criterion since $y^p$ is prime in $\mathbb{F}[x,y^p]$. 

So $[K(x,y):K(x)]=p$ and we conclude that $[L:K]=p^2$. 

Remember that an extension is separable if all its elements are separable. Now, neither $x$ nor $y$ are separable over $K$. Indeed since their minimal polynomials are over a field of characteristic $p$: $q_x(t)=(t-x)^p\in L[t]$, similarly for $q_y(t)$.

\item Since $[L:K]=p^2$, if a primitive element for $L$ exists, its minimal polynomial over $K$ must be of degree $p^2$. \\
On the other hand, let $\alpha\in L$ be any element. By definition of $L$ we can write $\alpha=\tfrac{f}{g}$ with $f,g\in \mathbb{F}_p[x, y]$. 

Observe that for any $f\in \mathbb{F}_p[x, y]$,  $f=\sum_{i,j}a_{ij}x^{i}y^{j}$, with $a_{ij}\in \mathbb{F}_p$. Again since we are working over a field of characteristic $p$: 
$$ f^p= \Big(\sum_{i,j}a_{ij}x^{i}y^{j}\Big)^p=\sum_{i,j}a_{ij}(x^p)^{i}(y^p)^{j}\in \mathbb{F}_p[x^p, y^p]$$.

Hence we have shown that for every $f \in \mathbb{F}_p[x, y]$, $f^p\in \mathbb{F}_p[x^p, y^p]$. But this means that for every $\alpha \in L$, 
$$\alpha^p = (\frac{f}{g})^p =\frac{f^p}{g^p} \in \mathbb{F}_p(x^p, y^p)=K$$

Hence we have shown that for every $\alpha \in L$, $\alpha^p\in K$. But then $\alpha$ is root of the polynomial $p(t)=t^p-\alpha^p \in K[t]$. So the minimal polynomial of all elements of $L$ has degree smaller than $p$, thus the extension $L$ has no primitive element. 

\end{enumerate}


\end{document}

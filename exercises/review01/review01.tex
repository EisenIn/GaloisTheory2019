
\documentclass[12pt,a4paper]{article}
\usepackage{fancyhdr}
 
\pagestyle{fancy}
\fancyhf{}
\lhead{Math-317 Galois Theory \\ Review Problems 1}
\rhead{October 15, 2019}
\rfoot{Problems selected by J.Baudin, F. Eisenbrand S.Giampietro and V.A.Nadarajan}

\usepackage{times}
\usepackage{mathptmx}
\usepackage{mathrsfs}
\usepackage{amssymb}
\usepackage{amsmath}
\usepackage{enumerate} 
\usepackage{../../utf8math}


\DeclareMathOperator{\gal}{Gal}

\begin{document}




\vskip 2ex

\noindent
Hand in your handwritten and carefully prepared solutions on Wednesday, October 23, during the exercise session.

\vskip 2ex

\begin{enumerate}
\item Let $E \supseteq F$ be fields and $u,v \in E$. Show the following.  If $v$ is transcendental over $F$ but algebraic over $F(u)$, show that $u$ is algebraic over $F(v)$. 
\item Let $F$ be a field of characteristic $p>0$ and $K \supseteq F$ an algebraic extension. We say that $u\in K$ is \textit{purely inseparable} over $F$ if its minimal polynomial of $u$ over $F$ has only one root in $\overline{F}$, an algebraic closure of $F$. Similarly we say that $K \supseteq F$ is a \textit{purely inseparable extension} if every $u \in K$ is purely inseparable over $F$. 
\begin{enumerate}[a)]
\item Show that $K \supseteq F$ is a purely inseparable extension if and only if for every $u \in K$ there exists $d\geq 0$ such that $u^{p^d}\in F$ 
\end {enumerate}
Again consider $\overline{F} \supseteq F$ its algebraic closure, and let $\overline{F}^{\text{insep}} \subset \overline{F}$ be the set of purely inseparable elements of $\overline{F}$. 
\begin{enumerate}[b)]
\item Show that $\overline{F}^{\text{insep}}$ is a sub-field of $\overline{F}$. 
\end{enumerate}
\begin{enumerate}[c)]
\item Show that $\overline{F}^{\text{insep}}$ is a perfect field and is the minimal algebraic extension over $F$ in $\overline{F}$ that has this property (i.e. if $\overline{F} \supseteq L \supseteq F$ and $L$ is a perfect field, then $L\supseteq\overline{F}^{\text{insep}}$).
\end{enumerate}


\item Let $m,n\in\mathbb{N}$. Consider the polynomials $x^m-1$ and $x^n-1$ $\in \mathbb{C}[x]$, let $p(x)$ be their g.c.d. Show the following statements:
      \begin{enumerate}
          \item $p(x)\in\mathbb{Z}[x]$.
          \item $p(x)=x^d-1$ where $d=g.c.d.(m,n)$ [Hint: You may like to compute the g.c.d. in a field extension where $x^m-1$ and $x^n-1$ split.]
      \end{enumerate}{}
  
    \item Let $α$, $β \in \mathbb{C}$ be algebraic numbers (i.e. algebraic over $\mathbb{Q}$) and let $m_α(x)$, $m_β(x) \in \mathbb{Q}[x]$ be their respective  minimal polynomials and suppose that both have degree at most $d$. 
      Let $P_n$ denote the probability that, if we pick an integer $m \in \{1, \dots, n\}$ uniformly at random, then  $\mathbb{Q}(α, β) = \mathbb{Q}(α + mβ)$. Show that \[ P_n \geq 1 - \frac{d^2}{n} \] 

\end{enumerate}
\end{document}

%%% Local Variables:
%%% mode: latex
%%% TeX-master: t
%%% End:

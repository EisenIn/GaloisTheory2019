\documentclass[12pt,a4paper]{article}
\usepackage{fancyhdr}
\usepackage{amssymb}
\usepackage{amsmath, amsthm}
\usepackage{xfrac}    
\usepackage{tikz-cd}
\usepackage{enumitem}
\usepackage{commath}
 
\pagestyle{fancy}
\fancyhf{}
\lhead{Math-317 Galois Theory \\ Problem Set 5}
\rhead{October 16, 2019}
\rfoot{Problems selected by J.Baudin, F. Eisenbrand S.Giampietro and V.A.Nadarajan} 


\newcommand{\Q}{\mathbb{Q}}
\newcommand{\F}{\mathbb{F}}
\newcommand{\Lbar}{\overline{K}}

\usepackage{times}
\usepackage{mathptmx}
\usepackage{mathrsfs}
\usepackage{amssymb}
\usepackage{amsmath}
 
\usepackage{../../utf8math}


\DeclareMathOperator{\gal}{Gal}


\begin{document}


\vskip 2ex

\begin{enumerate}
\item 
Let $K$ be a field and $\Lbar$ be an algebraic closure of $K$. Furthermore, let $\Lbar \supseteq L \supseteq K$ where $L \supseteq K$ is a finite field extension of $K$. The goal of this exercise is to show the following : the number of field homomorphisms $L \rightarrow \Lbar$ that extend a given field homomorphism $\varphi : K \rightarrow \Lbar$ is at most $n = [L : K]$.
\begin{enumerate}
	\item Solve the problem if $L = K(a)$ for some $a \in L$ (i.e. the extension $L \supseteq K$ is simple). \\
	\emph{Hint : You may want to see what happens when $\varphi$ is just the standard inclusion.}
	\item Now suppose suppose that the extension is not simple. Proceed by induction on $n$ for any field extension to prove the result. 
	\\ \emph{Hint : Consider the tower $K \subsetneq K(a) \subsetneq L$ for some $a \in L \setminus K$}.
\end{enumerate}

\item 
Let $E ⊇ F$ a finite field extension and $G$ its Galois group. Show that $E ⊇F$ is Galois if and only if $|G| = [E:F]$. 

\item 
Let $p$ be a prime number, $q = p^m$  for some $m \geq 1$ and consider the extension $\F_{q^n} \supseteq \F_q$. Show that this extension is normal and separable. \\

\item 
Let $a_1, \dots, a_n$ be pairwise coprime integers different than $1$ that are squarefree (i.e. they are not divisible by $p^2$ for any prime $p$). The goal is to show that $[ \Q(\sqrt{a_1}, \dots, \sqrt{a_n}) : \Q] = 2^n$.
\begin{enumerate}
	\item Solve the problem for $n = 1$.
	\item Let $n > 1$ and suppose that the result holds for any $1 \leq j < n$. Show that the problem reduces to show that $\sqrt{a_n} \notin \Q(\sqrt{a_1}, \dots, \sqrt{a_{n - 1}})$.
	\item Suppose by contradiction that this does not hold. Then there exist $a, b \in \Q(\sqrt{a_1}, \dots, \sqrt{a_{n - 2}})$ such that \[\sqrt{a_n} = a + b\sqrt{a_{n - 1}} \] Prove that either $a$ or $b$ must be $0$. Furthermore, derive a contradiction if $b = 0$.
	\item Therefore, $a = 0$. Find a squarefree integer $c$ that is coprime to $a_1, \dots, a_{n - 2}$ such that $\sqrt{c} \in \Q(\sqrt{a_1}, \dots, \sqrt{a_{n - 2}})$ and derive a contradiction form this fact. \\
\end{enumerate}

\end{enumerate}
\end{document}

%%% Local Variables:
%%% mode: latex
%%% TeX-master: t
%%% End:

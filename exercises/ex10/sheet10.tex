\documentclass[12pt,a4paper]{article}
\usepackage{fancyhdr}
 
\pagestyle{fancy}
\fancyhf{}
\lhead{Math-317 Galois Theory \\ Problem Set 10 }
\rhead{November 19, 2019}
\rfoot{Problems selected by J. Baudin, F. Eisenbrand, S. Giampietro and V.A.Nadarajan}

\usepackage{amssymb}
\usepackage{amsmath, amsthm}
\usepackage{xfrac}    
\usepackage{tikz-cd}
\usepackage{enumitem}
\usepackage{commath}

\begin{document} 
\noindent


\begin{enumerate}
\item 
Let E be a finite Galois extension of F with Galois group G, and let L be the fixed field of a subgroup H of G i.e. $L=H^*$. Show that the automomorphism group of L $Gal(L:F) \cong N/H$ where N is the normalizer of H in G. (This is a refinement of Sheet 9 Ex 2).

\item 
Let $K$ be a field. We will call the Galois group of the splitting field of a separable polynomial $f\in K[x]$ as the Galois group of $f$ over $K$ and denote it by $G_f$. (if $K$ is perfect every polynomial in $K[x]$ has a Galois group associated to it.)
\begin{enumerate}
    \item Show that the Galois group of $x^3-3x+1\in\mathbb{Q}[x]$ is the alternating group $A_3$.
    \item Show that the Galois group of $x^3+3x+1\in\mathbb{Q}[x]$ is the symmetric group $S_3$.
    \item Find an algorithm to compute the Galois group (over $\mathbb{Q}$) of a monic cubic polynomial with coefficients in $\mathbb{Z}$. 
\end{enumerate}{}

\item 
We state a theorem which we will not prove: Recall that (with the notation above) $G_f$ acts on the roots of the polynomial $f$. If we denote $deg(f)=m$, this action induces (up to conjugation) an embedding of $G_f$ in $S_m$.\\

\textbf{Theorem [Dedekind]}\\

Let $f\in\mathbb{Z}[x]$ be a monic polynomial of degree m, and let p be a prime such that $f \mod p$ has simple roots. Suppose that $f=\Pi_{i=1}^{r} f_i \in \mathbb{F}_p[x]$ with $f_i\in\mathbb{F}_p[x]$ irreducible of degree $m_i$. Then the image of $G_f$ in $S_m$ contains an element whose cycle decomposition is of type $(m_1,...,m_r)$.\\

\begin{enumerate}
    \item Show that for $f$ as above $D(f)\in\mathbb{Z}$, $f \mod p$ has simple roots iff $p \nmid D(f)$, where $D(f)$ is the discriminant of $f$. This shows that the theorem of Dedekind applies to all but finitely many primes. \\ \\
    \emph{Hint : You can apply the following result without proof : let $R$ be a ring and let $f \in R[x_1, \dots, x_n]$ be a symmetric polynomial (i.e. $f(x_{\sigma(1)}, \dots, x_{\sigma(n)}) = f(x_1, \dots, x_n)$ for all $\sigma \in S_n$). Then $f \in R[s_1, \dots, s_n]$ where the $s_i$'s are the elementary symmetric polynomials which you have seen in the course.}
    \item Show that the Galois group of $X^5-X-1$ (over $\mathbb{Q}$) is $S_5$. [Hint: Consider factorisations of this polynomial modulo small primes.]
    \item Show that a transitive subgroup of $S_n$ containing a $n-1$-cycle and a transposition is $S_n$. (Recall that a transitive subgroup is a subgroup s.t. its action on $\{1,...,n\}$ is transitive.)
    \item Use the above fact and the theorem of Dedekind to give a construction of a polynomial of degree $n$ with Galois group (over $\mathbb{Q}$) $S_n$ for $n\geq 1$. Note that this shows that the upper bound on the degree of the splitting field over $\mathbb{Q}$ is tight. 
    
\end{enumerate}

\textbf{Remark}

The theorem of Dedekind plays a key role in algorithms to calculate Galois group of monic polynomials with integral coefficients. If you are interested in learning more, you can look at \textbf{L. Soicher, J. McKay, Computing Galois groups over the rationals, J. Number Theory 20 (1985), 273–281}.

\item Let $\phi \colon G \rightarrow G'$ be a surjective group homomorphism.
  \begin{enumerate}[label = \alph*)]
  \item If $N $ is a normal subgroup of $G$ then $\phi(N)$ is a normal subgroup of $G'$.
  \item If $N'$ is a normal subgroup of $G'$ then $\{ g \in G \colon \phi(g) \in N'\}$ is a normal subgroup of $G$. 
  \end{enumerate}

\item
  Let $G$ be a group and $M$ be a subgroup of $G$. Then $M$ is an inclusion-wise maximal normal subgroup of $G$ if and only if $G/M$ is simple.

  \emph{Recall that a group $G$ is simple if the only normal subgroups of $G$ are $\{1\}$ and $G$.}
\item Show that a finite group $G$ is solvable, if and only if there are subgroups
  \begin{displaymath}
    \{1\} = G_n \subseteq G_{n-1} \subseteq \cdots \subseteq G_1 \subseteq G_0 = G  
  \end{displaymath}
  such that
  \begin{enumerate}[label = \alph*)]
  \item $G_i$ is normal in $G_{i-1}$
  \item $|G_{i-1}| : |G_i|$ is a prime. 
  \end{enumerate}
\end{enumerate}
\end{document}
%%% Local Variables:
%%% mode: latex
%%% TeX-master: t
%%% End:
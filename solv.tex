\chapter{Applications}
\label{cha:applications}


\section{Insolvability of polynomials}
\label{sec:insolv-polyn}

In this section, we will show a connection between the Galois group of a polynomial and the solvability of a polynomial in radicals.

\begin{definition}
  \label{def:6}
  Let $f(x) ∈ F[x]$ be a polynomial, where $F$ is a field. The \emph{Galois group} of  $f(x)$ is $\gal(E:F)$, where $E$ is a splitting field of $f(x)$. 
\end{definition}

This definition makes sense because if $E'$ is another splitting field of $F$, then $E$ and $E'$ are isomorphic and therefore, also $\gal(E:F)$ and $\gal(E':F)$ are isomorphic. (Exercise)


\begin{definition}
  \label{def:7}
  A field extension  $E ⊇F$  is called \emph{radical extension}, if a chain of intermediate fields  $E = E_0 ⊇ E_1 ⊇ \cdots ⊇ E_n = F$ exists, such that
  \begin{displaymath}
    E_i = E_{i+1}(u_i), \text{ where } u_i^{n_i} ∈ E_{i+1} \text{ for some } n_i≥1. 
  \end{displaymath}
  A polynomial $f(x) ∈F[x]$ is called \emph{solvable} if some radical extension of $f(x)$ contains a splitting field of $f(x)$. 
\end{definition}
Thus, if a polynomial is solvable, then each of its roots can be written as a nested finite  expression involving elements of $F$, addition, subtraction, multiplication and division, as well as taking $n$-th roots.
A rational polynomial of degree at most $4$ is solvable. 
Our goal is to show that there exists a  polynomial $f(x) ∈ℚ[x]$ of degree $5$ that is not solvable. 
%
We next recall the definition and some characterization of a solvable group. 
\begin{definition}
  \label{def:8}
  A group  $G$ is \emph{solvable} if there exists a chain of subgroups
  \begin{displaymath}
    G = G_0 ⊇ G_1 ⊇ \cdots ⊇ G_n = \{1\}
  \end{displaymath}
  such that
  $G_{i+1} ◁ G_i$ and $G_i / G_{i+1}$ is abelian for each $1≤i <n$. 
\end{definition}

\begin{lemma}
  \label{lem:7}
  If $G$ is a group and $K ◁ G$, then $G$ is solvable if and only if $K$ and $G/K$ are solvable. 
\end{lemma}

We will show the if direction of the Galois criterion. This will enable us to show that a certain polynomial is not solvable by showing that its Galois group is not solvable. 
\begin{theorem}[Galois criterion] 
  \label{thr:26}
  Let $F$ be a field of characteristic zero and $f(x) ∈ F[x]$. The Galois group of $f(x)$ is solvable if and only if $f(x)$ is a solvable polynomial. 
\end{theorem}
The group $S_5$ is not solvable. We will construct a polynomial of degree $5$ in $ℚ[x]$ whose Galois group is $S_5$.

\begin{lemma}
  \label{lem:8}
  If $p$ is a prime number, then $S_p$ is generated by any $2$-cycle, together with any $p$-cycle. 
\end{lemma}


\begin{proof}
  We assume that the $p$-cycle is $ σ= (1,2, \dots,p)$ and the $2$-cycle is $τ=(1,k)$. Now $σ^{k-1}$ is  a $p$-cycle as well that maps $1$ to $k$. Therefore, we may assume that $ σ= (1,2,\dots,p)$ and $τ = (1,2)$. We have $(k+1,k+2) = σ^k τ σ^{-k}$. Since $(1,a+1) = (1,a) (a,a+1)(1,a)$ it follows that, for each $a$,  $(1,a)$ is in the subgroup generated by $τ$ and $σ$. The transpositions $(1,a)$ generate $S_p$. 
\end{proof}

\begin{example}
  \label{exe:10}
  The polynomial $p(x) = x^5 - 6x +2 ∈ ℚ[x]$  is not solvable. First of all, we note that $p(x)$ is irreducible (Eisenstein). Let $E$ be the splitting field of $p(x)$. The Galois group $G =\gal(E:ℚ)$ is isomorphic a subgroup of the permutations of the roots $X ⊆ ℂ$ of $p$.

  The polynomial $p(x)$ has three distinct real roots. This can be seen by inspecting the real roots of the derivative $p'(x) = 5x^4 -6$, which are $\pm \sqrt[4]{6/5}$. The polynomial is positive at $-\sqrt[4]{6/5}$ and negative at $\sqrt[4]{6/5}$. The other two non-real roots are complex conjugates of each other. Convex conjugation is in $G$, it is a transposition of the complex roots.

  Since $E⊇ ℚ$ is Galois, we have $ |G| = [E:ℚ]$. Since $p$ is irreducible and of degree $5$, we have that $5$ divides $G$ and thus there exists an element of order $5$ in $G$ by Cauchy's Theorem. The only elements of order $5$ in $S_5$ are the $5$-cycles. Thus, by Lemma~\ref{lem:8}, $G$ is isomorphic to $S_5$ and thus not solvable. By the Galois criterion, $p(x)$ is not solvable. 
\end{example}


We will now prepare for the proof of the \emph{if} direction of the Galois criterion, namely, if $f(x)$ is solvable, then its Galois group is a solvable group.

\begin{lemma}
  \label{lem:9}
  If $G$ is a cyclic group of order $n$, then $\mathrm{aut}(G)≅ ℤ_n^*$. 
\end{lemma}

Let $E ⊇F$ be a field extension. An element  $ω ∈ E$ is a \emph{primitive $n$-th root of unity over $F$} if it is a root of $x^n -1$ and if its order is $n$. 

\begin{theorem}
  \label{thr:28}
  If $F$ is a field and $n≥1$ is an integer, then a primitive $n$-th root of unity $ω$ over $F$ exists if and only if $\car(F)$ does not divide $n$. In this case,
  \begin{enumerate}[i)]
  \item $F(ω)$is the splitting field of $x^n-1$ over $F$. \label{item:11} 
  \item $F(ω) ⊇ F$ is a finite Galois extension and $\gal(F(ω):F)$ is isomorphic to a subgroup of $ℤ_n^*$. 
  \end{enumerate}
\end{theorem}


\begin{theorem}
  \label{thr:29}
  Let $F$ be a field containing a primitive $n$-th root of unity and consider an extension $F(u) ⊇ F$,where $u^n ∈ F$. Then $F(u) ⊇ F$ is a Galois extension and $\gal(F(u):F)$ is abelian. 
\end{theorem}


\begin{theorem}
  \label{thr:30}
  Let $E ⊇F$ be a radical Galois extension, where $\car(F) = 0$, then  $\gal(E:F)$ is a solvable group. 
\end{theorem}
%%% Local Variables:
%%% mode: latex
%%% TeX-master: "notes"
%%% End:
